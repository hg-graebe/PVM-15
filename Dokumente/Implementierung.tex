\documentclass[a4paper,11pt]{article}
\usepackage[utf8]{inputenc} 
\usepackage{a4wide,url,ngerman,graphicx}

\parindent0pt
\parskip3pt
\setcounter{tocdepth}{2}

\title{Partizipatorisches Virtuelles Museum\\[.6em] Implementierungsdetails}
\date{Version vom 23. September 2016}
\author{Sebastian Günther, Hans-Gert Gräbe}

\begin{document}
\maketitle
\tableofcontents
\newpage
 
\section{Allgemein}
 
Das \emph{Partizipatorische Virtuelle Museum} (PVM) ist als digitale
Online-Plattform konzipiert, in der sich die Besucher mit Gegenständen aller
Art künstlerisch auseinandersetzen können. Dazu sollte auf Basis von Wordpress
in Kombination mit einer eigenen Plugin-Suite und ausgewählten
Drittanbieter-Plugins in mehreren Iterationsstufen eine möglichst robuste
Lösung entwickelt werden, die moderne Designansätze wie \emph{responsive
  Design} umsetzt und auch in Zukunft Spielraum für Erweiterungen lässt.
Zusätzlich sollte damit der Bedienkomfort der Seite erhöht und der
Wartungsaufwand minimiert werden.

Eine erste Version der
Plattform\footnote{\url{http://pcai042.informatik.uni-leipzig.de/swp/SWP-15/index.html},
  Gruppe SWP-15.} wurde im Sommersemester 2015 im Rahmen des SWT-Praktikums am
Institut für Informatik der Uni Leipzig entwickelt.  Die vorliegende Version
entstand im Studienjahr 2015/16 als gründliches Redesign dieser Basisversion,
wobei detaillierte Gestaltungsanforderungen aus einem Begleitseminar am
Institut für Kunstpädagogik (Leitung Hendrikje Düwel) einbezogen und umgesetzt
wurden.  Der Designentwurf und die Implementierung wurden überwiegend von
Sebastian Günther ausgeführt. Zeitweise beteiligt waren Alexander Lieder und
Maximilian Malios. 

Die folgende Beschreibung setzt die Kenntnis der grundlegenden Konzepte aus der
Projekt-Beschreibung, siehe \texttt{beschreibung.pdf}, sowie der Terminologie
im Umfang des dort zusammengestellten Glossars voraus.

\section{Produktübersicht}

\subsection{Prinzipielle Architektur-Entscheidungen}
 
Basis der Plattform ist eine Wordpress-Single-Site-Installation, deren Look and
Feel durch Auswahl und nachfolgende Anpassung eines geeigneten WP-Themes für
den Anwendungszweck gestaltet wurde.  Zur Trennung von Layout und
Funktionalität wurde ein eigenes Plugin \texttt{pvmkit} entwickelt, in dem die
PVM-spezifischen funktionalen Aspekte gekapselt sind. Für Stellen, wo eine
Verzahnung von Funktion und Layout erforderlich ist (etwas Such- oder
Übersichtsseiten), wurde das von der Designgruppe ausgewählte Basisthema
\emph{esteem} mit einem Child-Theme \texttt{esteem\_pvm\_child} modifiziert. 

\subsection{Workflow aus Nutzersicht}

Auf der Startseite, auf die man über den Button „Me \& Things“ zurückgelangt,
werden neben dem Navigationsmenü eine Übersicht über die verfügbaren Werke
sowie die vorgegebenen Merkmale angezeigt. Das Navigationsmenü ist als
Zwei-Ebenen-Menü ausgestaltet und erlaubt eine grundsätzliche Navigation
zwischen den einzelnen Bereichen \emph{About}, \emph{Tutorial},
\emph{Inspiration} und \emph{Me Projects} der Plattform.

Die Navigation zwischen verschiedenen Werken (oder verschiedenen Projekten, to
do) ist als Kachelnavigation auf der Startseite ausgestaltet, wobei durch
entsprechende Filterkriterien die Anzahl der angezeigten Werke eingeschränkt
werden kann.

In der Fußleiste sind Links auf \emph{Impressum}, \emph{AGB}, \emph{FAQ} und
\emph{Login} vorhanden.  

Als nicht angemeldeter Betrachter hat man Zugriff auf die Filterfunktion, mit
deren Hilfe man die Anzahl der gelisteten Werkkacheln nach den vorgegebenen
Merkmalen einschränken kann. Außerdem kann man auf den einzelnen Werkseiten
die jeweiligen Ausstellungsstücke studieren. Weiterhin kann man sich an der
Plattform registrieren oder anmelden.
 
Angemeldete Benutzer (User) können weitere Funktionen der Plattform nutzen.
Jegliche Partizipation (wie Einstellen neuer Werke oder Bewertung vorhandener)
setzt eine Anmeldung voraus.

\section{Grundsätzliche Struktur- und Entwurfsprinzipien}
 
Zur Umsetzung der Funktionalität wurde ein WP-Plugin \texttt{pvmkit} erstellt.
Aus Übersichtlichkeitsgründen und um die Updatefähigkeit von Wordpress zu
wahren, benutzt \texttt{pvmkit} ein eigenes Datenmodell, das die
Wordpressdatenbank um mehrere Tabellen erweitert.  
 
Das Plugin \texttt{pvmkit} hat eine eigene Verzeichnis- bzw. Dateistruktur. Im
Hauptverzeichnis des Plugins befinden sich die Dateien \texttt{pvmkit.php}, und
\texttt{functions.php}, in den Unterverzeichnissen \texttt{backend},
\texttt{classes}, \texttt{images}, \texttt{libs}, \texttt{shortcodes} und
\texttt{widgets} verschiedene funktionale Einheiten, die weiter unten genauer
beschrieben werden und vor allem die Applikationslogik sowie die Anbindung der
Datenschicht realisieren.

View-Funktionalitäten, insbesondere die für verschiedene Zwecke erforderlichen
Seitentemplates, sind im Child Theme \texttt{esteem\_pvm\_child} zu finden. Die
Templates greifen dabei auf die im Plugin implementierten Funktionalitäten
zurück.  Die genauen Abhängigkeiten können durch die im Kopfbereich des
jeweilgen Templatecodes inkludierten Dateien nachvollzogen werden.

Weiterhin wird das Verzeichnis \texttt{wp\_content/uploads/pvm} der
WP-Installation zur Ablage von Paketinformationen zu den einzelnen Werken
verwendet.

Die Datei \texttt{pvmkit.php} enthält die Plugin-Definition und dient als
Container, über den die Funktionalitäten aus den einzelnen Unterverzeichnissen
(Paketen) eingebunden werden. 

Die Benutzerregistrierung und Authentifizierung wird derzeit über das Plugin
\texttt{wp-members} realisiert, das dafür entsprechende Shortcodes
bereitstellt. 
 
\section{Struktur- und Entwurfsprinzipien einzelner Pakete}
 
Strukturierung des Codes wird durch Unterverzeichnisse erreicht: 
\begin{itemize}\itemsep0pt
\item \texttt{backend} -- Enthält die Einbindung der Werkadministration ins
  Backend. 
\item \texttt{classes} -- Enthält alle Klassendefinitionen und
  Vererbungsstrukturen. 
\item \texttt{images} -- (Funktionale) Bilddateien, die an verschiedenen
  Stellen eingebunden werden.
\item \texttt{libs} -- PHP-Bibliothek \emph{EasyRDF}.  Diese wird in
  \texttt{pvmkit\_package\_editable.php} zum Parsen der Manifest-Datei eines im
  PVM-Paketformat als zip-Datei vorliegenden Werks benötigt.  
\item \texttt{shortcodes} -- Verschiedene WP-Shortcodes. 
\item \texttt{widgets} -- Widgets für die Sidebar.
\end{itemize}

\subsection{Erstellen und Verwalten von Werken}

Relevante Datenbanktabellen sind    
\begin{itemize}\itemsep0pt
\item \texttt{pvmkit\_packages}: Tabelle der Werke. Jedes Werk hat u.a. die
  Felder id, title, author (Verweis in die Autorentabelle),
  Veröffentlichungsdatum und Status.  Über den Status (status=8 bedeutet „Werk
  veröffentlicht“) wird der Werk-Workflow überwacht.  
\item \texttt{pvmkit\_package\_components}: Zuordnung von Id und Typ der
  einzelnen Werkteile zu einem Werk. Das dem Werk zugeordnete Titelbild wird in
  der Werkübersicht als Kachel dargestellt und dient dort der Navigation auf
  die Werkeinzelseite.
\item \texttt{pvmkit\_objects}: Daten der einzelnen Werkteile.               
\item \texttt{pvmkit\_object\_properties}: -- Tabelle ist obsolet.
\item \texttt{pvmkit\_property\_index}: Zuordnung der Eigenschaften zu einem
  Werk.    
\end{itemize}

\subsubsection*{Workflow Werke}

Ein Werk wird grundsätzlich über den ins Frontend integrierten Editor
(Edit-Template) erstellt, der auf der Profilseite des Autors verlinkt ist.
Über die Einstiegsseite des Editors können die verschiedenen Bestandteile
eines Werks hochgeladen (Bilder) bzw. erstellt (Text) sowie Merkmale
zugeordnet werden.  Weiterhin können fertige Werke zur Freigabe gemeldet
werden. 

Im Zuge des Editierprozesses werden
\begin{itemize}\itemsep0pt
\item Eine Werk-Id vergeben und ein Zustandsflag gesetzt.
\item Werkdaten in die Datenbank eingetragen.
\item Bilder entsprechend zugeschnitten und abgespeichert.
\item Der Text in die Datenbank übernommen.
\item Merkmale entsprechend der hinterlegten morphologischen Tabelle zugeordnet.
\item Das Autorenprofil zugeordnet. Über die Autoren-Id ist eine User-Id zu
  erreichen, wenn der Autor als User an der Plattform existiert.  
\item Werke sind in dieser Phase im Zustand 1.
\item Alle Werke eines Autors, auch die im Zustand 1, sind über die
  Profilseite des Autors unter dem Link „Eigene Werke“ gelistet.
\item Mit einem Link „Editieren“ kann das Edit-Formular gestartet werden.
\item Mit einem Link „Veröffentlichen“ wird  das Werk zur Veröffentlichung
  eingereicht wird und geht damit in den Zustand 5 über. 
\item Mit einem Link „Löschen“ wird das Werk in den Zustand 0 versetzt (das
  Werk ist noch da, aber als gelöscht markiert). 
\end{itemize}
Alternativ erhält man über den Link „Eigene Werke“ auf der Profilseite des
Autors eine Auflistung aller eigenen Werke im Kachelformat. Jedes noch durch
den Autor editierbare Werk (also im Zustand 1) het einen Edit-Button, über den
man ebenfalls auf die Einstiegsseite des Editors gelangt und das Editieren des
Werks fortsetzen kann.  Hier lässt sich auch das Hochladen eines Werks im
als zip-Datei PVM-Paketformat einbinden (aktuell nicht implementiert). 

Im Zustand 5 kann das Werk nicht mehr vom Autor editiert werden. Das Werk
erscheint in der Redaktionsliste „Werke zur Freigabe“ und kann von einem
Redakteur
\begin{itemize}\itemsep0pt
\item über den Link „Vorschau“ in einer Vorschau betrachtet werden (dazu wird
  das Seitentemplate „Einzelnes Werk“ mit der jeweiligen Werk-Id aufgerufen), 
\item über den Link „Werk editieren“ redaktionell überarbeitet werden (dazu
  wird das Edit-Template für Werke gestartet),
\item über den Link „Freischalten“ in den Fundus übergeben werden, womit das
  Werk in der Zustand 8 übergeht und in diesem Zustand weder vom Autor noch von
  der Redaktion bearbeitet werden kann, 
\item über den Link „Löschen“ in den Zustand 0 (Werk als gelöscht markiert)
  überführt werden.
\end{itemize}
Ein Werk im Zustand 0 kann komplett aus der Datenbasis gelöscht werden („delete
permanently“)\footnote{Siehe dazu die Funktion \texttt{delete\_package\_db}, in
  der diese Funktionalität über einen zweiten Parameter rudimentär
  implementiert ist.}.  Bis dahin ist das Werk noch über seine Werk-Id zu
erreichen und kann durch unmittelbare Modifikation des Status-Flags in der
Datenbank wieder aktiviert werden.

Ergänzungen zum Workflow Werke
\begin{itemize}\itemsep0pt
\item Bedeutung Werk-Status: 0=gelöscht, 1=wird erstellt, 5=eingereicht,
  8=freigegeben.
\item Auch erst teilweise fertiggestellte Werke sind ablegbar, wenn die
  erforderlichen weiteren Elemente gerade nicht zur Hand sind oder im Rahmen
  der Schulstunde nicht hochgeladen werden können.  Das Werk bleibt im
  Zustand~1. 
\item Der Autor findet auf seiner Profilseite in der Übersicht „eigene Werke“
  auch all seine Werke „in Bearbeitung“.
\end{itemize}

\subsection{Erstellen und Verwalten von Projekten}
 
Relevante Datenbanktabellen sind    
\begin{itemize} \itemsep0pt
\item \texttt{pvmkit\_projects} -- Ein Datensatz dieser Tabelle repräsentiert
  ein Projekt, welches von einem User initiiert wurde.
\item \texttt{pvmkit\_project\_participants} -- Besitzer des Projekts. 
\item \texttt{pvmkit\_projects\_packages} -- Dem Projekt zugeordnete Werke.
\item \texttt{pvmkit\_project\_images} -- Dem Projekt zugeordnete Bilder.  Das
  dem Projekt zugeordnete Titelbild wird in der Projektübersicht als Kachel
  dargestellt und dient dort der Navigation auf die Projektseite.
\end{itemize}
 
Workflow Projekte (Anforderungen)
\begin{itemize}\itemsep0pt
\item Dieser soll ähnlich aufgebaut werden wie Workflow Werke: Von einer
  Hauptseite wird auf Seiten für die einzelnen Unteraufgaben (Text eingeben,
  Bild hochladen usw.) weitergeleitet und von dort zurück auf die Hauptseite.
\item Die Projektseite enthält drei Texte (Kurztext, Projektdokumentation,
  Abschlusstext), welche vom Projektleiter bereitgestellt und während des
  Projektverlaufs überarbeitet werden können; (+ dokumentarische Bilder)
\item Ähnlich wie bei den Favoriten wird auf der Profilseite eine Übersicht
  über die Projekte eines Nutzer angelegt, in denen er der Projektleiter ist.
\item Wie bei Werken muss auch ein Projekt in \emph{mehreren} Schritten mit
  Abspeicher\-möglichkeit des aktuellen Stands erstellbar sein.
\item Der Projekt-Lebenszyklus wird wie bei Werken über eine Statusvariable
  verwaltet, wobei wenigstens die Zustände „noch nicht freigegeben“,
  „freigegeben für initiale Werkzuordnung“, „in Realisierung“, „archiviert“
  vorzusehen sind.
\end{itemize}

\subsection{Seitenaufbau}

Eine Reihe von (scheinbar) individuellen Seiten wie etwa die Profilseite eines
Autors werden direkt und nur aus Informationen aus der PVM-Datenbank
aufgebaut.  Dazu wird ein spezielles Seitentemplate mit einer speziellen Seite
im Wordpress verbunden, der die darzustellende variierende Information über
GET-Parameter weitergegeben wird.  

So stellt \texttt{single\_user/?author\_id=23} die Profilseite des Autors mit
der \texttt{id=23} dar.

\begin{itemize} \itemsep0pt
\item \texttt{single\_user} -- Profilseite eines Autors.
\item \texttt{profile\_package\_list} -- Einstiegsseite in die Verwaltung der
  Werke eines Autors.
\item \texttt{autor-favoriten} -- Favoriten des Autors.
\item \texttt{single\_package} -- Einzelseite eines Werks.
\end{itemize}

Auf diese Weise wird ein konsistentes und einheitliches Design der
Profilseiten der Autoren usw. als eine grundlegende Anforderung aus den
Gestaltungsentwürfen erzwungen.

\section{Datenmodell}
 
Das Standard-Wordpress-Datenmodell wird nicht verändert, um eine reibungslose
Funktion und Updatefähigkeit des Grundsystems sicherzustellen. Das
PVM-Datenmodell ist deshalb konzipiert als Erweiterung des WP-Datenmodells um
PVM-spezifische Begrifflichkeiten, die über drei Bindings -- (1) an die
Tabellenstruktur der SQL-Datenbank, (2) an einen eigenen Upload-Bereich im
Dateisystem des Webservers und (3) an die CSS-Struktur des verwendeten
WP-Basisthemes -- die entsprechenden WP-Datenrepräsentationen erweitert.

Die Erweiterung (1) der Datenbank erfolgt durch mehrere Tabellen, welche durch
das Präfix\footnote{Hinzu kommt das allgemeine WP-Datenbank-Präfix, mit dem
  verschiedene WP-Installationen unterschieden werden können, die eine
  gemeinsame Datenbank nutzen.} \texttt{pvmkit\_} gekennzeichnet sind. Neben
den oben bereits aufgelisteten Tabellen sind dies
\begin{itemize}\itemsep0pt
\item \texttt{pvmkit\_authors}: Tabelle der Zuordnung von Autoren zu
  WP-Usern.  

  Nur WP-User können als Autoren oder in Vertretung von Autoren agieren.  

  Ein User kann mehreren Autoren zugeordnet sein (Power-User\footnote{Etwa ein
  Tutor, der die Ergebnisse eines Offline-Kurses für die Autoren in die
  Plattform einspielt.}). 

  Das User-Feld kann auch leer sein. Dann ist der Autor nicht mehr an der
  Plattform aktiv.
\item \texttt{pvmkit\_notifications}: Tabelle mit Nutzerbenachrichtigungen. 
\item \texttt{pvmkit\_properties}: Tabelle der extern vorgegebenen
  Eigenschaften (morphologische Tabelle), die einzelnen Werken zugeordnet
  werden können.
\item \texttt{pvmkit\_ratings}: Bewertung der einzelnen Werke durch User.  
\end{itemize}

Das System der Benutzerrechte verwendet den ACL-Mechanismus von Wordpress; die
entsprechenden Rechte werden in der Tabelle \texttt{user\_meta} gespeichert. 

\section{Weitere Informationen}

\begin{itemize}\itemsep0pt
\item Projekten werden Werke, aber keine Teilnehmer mehr zugeordnet. Jeder
  Autor kann für jedes Projekt auf der Projektseite eigene Werke vorschlagen,
  die vom Projektleiter bestätigt werden müssen.
\item Ein User kann mehrere Autoren vertreten (Power-User). Dazu muss er vom
  Administrator auf seiner Profilseite das entsprechende Recht zugewiesen
  bekommen. 
\item Wenn ein User gelöscht wird, bleiben das oder die zugeordneten
  Autorenprofile bestehen.
\item Bereits freigegebene Werke können zurückgenommen werden (mehrfach von
  Frau Düwel praktizierter Use Case, weil die Werke den Qualitätsansprüchen
  nicht genügten). 

  Dazu müssten die Beteiligten entsprechend informiert werden.
\item Ein Power-User (mit mehreren Autorenprofilen) sieht im eingeloggten
  Zustand auf \emph{allen} Profilen \emph{alle} ihm zugeordtene Werke als
  „eigene Werke“.
\item Bilder zu Werken werden während des Erstellungsprozesses in verschiedenen
  Auf"|lösungen im Verzeichnis \texttt{uploads/pvm/images} abgelegt, Bilder zu
  Projekten im Verzeichnis \texttt{uploads/pvm/project\_images}.
\end{itemize}

\section{Offene Fragen}
 
\begin{itemize}\itemsep0pt
\item In einer frühen Planungsphase waren der Export eines Werks als zip-Datei
  im PVM-Paketformat (auf RDF-Basis) und die Möglichkeit des Hochladen eines
  solchen Werks vorgesehen.

  In der aktuellen Version wurde diese Variante des Persistierens nicht weiter
  entwickelt und steht damit auch nicht zur Verfügung.  Einzelne Dateien in den
  Verzeichnissen \texttt{uploads/pvm/archive} und \texttt{uploads/pvm/zip} auf
  dem Server sind noch Relikte dieser 

\item Im Plugin \texttt{pvm2rdf} werden prototypisch WP-Shortcodes definiert,
  mit denen sich Teile der PVM-spezifischen Tabellen als RDF ausgeben lassen.
  Siehe dazu auch den PHP-Code des Plugins. 

\item Volltextsuche über Werke einbinden.

\item Ansatz „social media“: Links sollen Share-Buttons sein wie aktuell
  bereits angelegt. Funktioniert bei Facebook nur zufällig. Nicht systematisch
  implementiert.
\end{itemize}

\end{document}
                 
