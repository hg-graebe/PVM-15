\documentclass[a4paper,11pt]{article}
\usepackage[utf8]{inputenc} 
\usepackage{a4wide,url,ngerman}

\newcommand{\Kommentar}[1]{
  \begin{quote}\textbf{Kommentar:} #1 \end{quote}
}

\newcommand{\FV}[1]{
  \begin{quote}\textbf{Formulierungsvorschlag für Einfügung:} #1 \end{quote}
}

\newcommand{\glossar}[1]{{$\to$ \textsc{#1}}}
\parindent0pt
\parskip3pt
\setcounter{tocdepth}{2}

\title{Partizipatorisches Virtuelles Museum\\[.6em] 
  Stand der Anforderungsaufnahme}
\date{Version vom 29. März 2016}
\author{Hans-Gert Gräbe}

\begin{document}
\maketitle

\tableofcontents

\newpage  

\emph{Ziel} dieses Dokuments ist es, den Stand der Anforderungserhebung zur
Erstellung und Weiterentwicklung der \glossar{Plattform} des PVM-Projekts
verbindlich festzuhalten.  Dieses Dokument wird durch Prof. Gräbe als
\emph{Project Owner} der IT-Gruppe fortgeschrieben und in regelmäßigen
Abständen mit Frau Düwel als \emph{Leiterin des Gesamtprojekts} entsprechend
den „Festlegungen zur Anforderungsaufnahme“ (siehe Kurs-Wiki) abgestimmt.

Das Dokument wurde zuletzt in einer Durchsprache am 14.1.2016 abgestimmt, im
Nachgang entsprechend der Durchsprachen zum Thema „Projekte“ modifiziert und
im Gespräch am 29.3.2016 ohne weitere Detaildurchsprache als Arbeitsgrundlage
bestätigt.

\section{Einführung}

Kern des Partizipatorischen Virtuellen Museums (PVM) ist eine digitale
\glossar{Plattform}, die auf Beteiligung von Internetusern ausgerichtet ist. Es
stellt eine virtuelle Ausstellungsform mit festgelegter Thematik und
Präsentationsart dar.  Die präsentierte \glossar{Ausstellung} entsteht jedoch
ausschließlich durch die \glossar{User} selbst.

Der Museumsbegriff wird im Kontext des PVM lediglich als \emph{Analogie}
bzw.\ Vergleich verwendet, um zu erklären oder Vorgänge zu beschreiben, die
einem Museum ähnlich sind, z.B. das Präsentieren von Objekten (etwa im
About-Text oder im Tutorial).  In der Außendarstellung und als Bezeichnung für
die Site soll der Museumsbegriff nicht verwendet werden, da Jugendliche und
junge Erwachsene dem \emph{traditionellen} Museum eher ablehnend
gegenüberstehen (vgl. Bocatius 2014).

\Kommentar{15.01.2015, Gräbe: Mit der Formulierung „Kern des \ldots“ wird dem
  Umstand Rechnung getragen, dass das PVM-Konzept über die Verfügbarkeit einer
  digitalen Plattform als \emph{virtuelle Ausstellungsform} hinausreicht und
  als „Museum“ weitergehende didaktische und konzeptionelle Ziele verfolgt,
  insbesondere mit der Vorgabe von Design-Richtlinien sowie der Möglichkeit zu
  \glossar{Projekten}. Diese weitergehenden Konzepte müssen im Lichte digitaler
  Technologien einzeln bewertet werden.  Das muss nicht unbedingt unter einem
  eigenen einheitlichen Begriff eines \emph{digitalen Museums} zusammengefasst
  werden. }

Auf der \glossar{Plattform} werden von \glossar{Usern} bereitgestellte digitale
Darstellungen von Gegenständen als \glossar{Werke} gesammelt. Die Gegenstände
werden repräsentativ als Darstellungsmittel der eigenen Person betrachtet.
Voraussetzung ist, dass der Gegenstand etwas mit der eigenen Person zu tun hat.
Dann kann er diese repräsentieren.

Der Schwerpunkt liegt auf der Sammlung und Reihung einer großen Anzahl von
\glossar{Werken}, die unterschiedlich zusammengestellt werden können, sowie auf
der interaktiven Teilnahme. 

\glossar{Werke} können von (nicht angemeldeten) \glossar{Besuchern} in
verschiedener Weise allein durch unterschiedliche Sortiermöglichkeiten
angeschaut werden.  Die Nutzung weitergehender partizipativer Möglichkeiten
setzt eine Registrierung an der Plattform als \glossar{User} voraus.

\glossar{Werke} sind im \glossar{PVM-Paketformat} für die Integration in die
\glossar{Plattform} über einen spezifizierten \glossar{Einreichungs}- und
\glossar{Freigabeprozess} bereitzustellen. Ein \glossar{Redakteur} entscheidet
über die Veröffentlichung.

Das Zusammenstellen von einzelnen \glossar{Werkteilen} zu einem \glossar{Paket}
im \glossar{PVM-Paket"|format} wird durch die Bereitstellung einer
\glossar{Formularseite} unterstützt.

Ein \emph{Ziel} des PVM-Projekts ist es, Lehrenden die Möglichkeit zu geben,
die \glossar{Plattform} im Unterricht zu nutzen.  Neben der Möglichkeit, auf
Inhalte über unsere allgemeinen Plattformkonzepte zuzugreifen, bieten wir
Lehrenden die Möglichkeit, eigene \glossar{Projekte} zu verwirklichen oder
eines unserer \glossar{Projektangebote} mit der eigenen Klasse umzusetzen.

Auf der Basis erster konzeptioneller Überlegungen zu einem
\emph{Partizipatorischen Virtuellen Museum} (PVM) von Frau Düwel wurde im
Sommersemester 2015 im Rahmen des Software"|technik-Praktikums eine erste
prototypische Software-Skizze eines solchen digitalen Museums auf der Basis von
Wordpress erstellt, die im Zuge folgender Semester im Rahmen eines Projekts der
Laboruniversität weiterentwickelt werden soll.

Grundlage der Arbeit und Orientierungsmuster waren neben diesen konzeptionellen
Überlegungen weiter verschiedene „Fundstücke“-Projekte am Institut für
Kunstpädagogik, die in den letzten Jahren aus studentischen Semesterarbeiten
entstanden sind.

\section{Konzeptionelle Grundlagen}

\subsection{Allgemeines}\label{grundlagen.allgemeines}

Das PVM-Konzept geht von der Grundvorstellung (Metapher) eines Museums aus, in
dem die öffentlich zugänglichen \glossar{Werke} als fertige, nicht weiter
modifizierbare Einheiten im \glossar{Fundus} als „Sammlung“ zusammengefasst
sind.  

Diese Werke sind in einer \glossar{Dauerausstellung} dem Publikumsverkehr
zugänglich und können von \glossar{Besuchern} über Such- und Filterfunktionen
auf der Basis der zugeordneten \glossar{Merkmale} verschieden strukturiert
werden.  Das Konzept der \glossar{Dauerausstellung} orientiert sich an der
Metapher einer großen Ausstellungshalle, in der die fertigen \glossar{Werke}
unterschiedslos platziert sind.

Derartige \glossar{Werke} werden in einem \glossar{Werkstattprozess} von
einzelnen Personen, den \glossar{Urhebern}, erstellt, nach der Erstellung
einem Begutachtungsprozess durch die \glossar{Redakteure} unterzogen und nach
einem positiven Votum in den \glossar{Fundus} übernommen werden.

\glossar{Werke} können ausschließlich in der \glossar{Werkstatt} bearbeitet
werden, außerhalb der Werkstatt sind sie unveränderlich.

Als dingliche Einheit liegen \glossar{Werke} in der Form von \glossar{Paketen}
zusammengehörender \glossar{Werkteile} vor, die zugleich
\begin{itemize}
\item Ergebnis des kreativen Schaffens von \glossar{Urhebern} im Sinne der
  PVM-Konzeption eines reflexiven Zugangs zum Selbst über dingliche
  Darstellungsmittel in privatem \glossar{Werkstattprozess} und
\item Ausgangspunkt rezeptiven Zugangs (anonymer) \glossar{Besucher} zu einer
  vernetzten Vielfalt solcher Artefakte in der \glossar{Dauerausstellung}
  sind.
\end{itemize}
Solche \glossar{Pakete} sind im \glossar{PVM-Paketformat} zum Import in die
\glossar{Plattform} bereitzustellen.

Registrierte Nutzer (\glossar{User}) können neue \glossar{Werke} hochladen, die
aus
\begin{itemize}\itemsep0pt
\item Titel, 
\item Titelbild als \glossar{Bild-Teil},  
\item einer Assoziation dazu als \glossar{Text-Teil}, 
\item einer bildnerischen Darstellung als weiterem \glossar{Bild-Teil} sowie
\item Merkmalszuweisungen aus einer vorgegebenen Menge von \glossar{Merkmalen}
\end{itemize}
bestehen.  Die \glossar{Merkmale} sind als morphologische Tabelle
plattformweit vorgegeben.  Zu jedem Werk ist weiter dessen \glossar{Urheber}
sowie das Einreichungsdatum hinterlegt.  Über den \glossar{Urheber} ist eine
Orts-Information verfügbar. Neue Werke können grundsätzlich nur registrierte
\glossar{User} einreichen.  Der Import von externen Paketen durch Dritte ohne
Beteiligung des Urhebers ist nicht vorgesehen.

In ausgebautem Zustand soll die inhaltliche Weiterentwicklung der Plattform
erfolgen 
\begin{enumerate}
\item über \glossar{Projekte}, die beide Aspekte (Reflexion in speziell
  konzipiertem \glossar{Ausstellungskontext} und private Kreativität in
  alters- und zielgruppenspezifisch angepasstem \glossar{Werkstattprozess})
  unter je spezifischen didaktischen Zielstellungen miteinander verbinden
  sowie 
\item über den musealen Allgemeinprozess in der \glossar{Dauerausstellung}
  (Einreichen neuer \glossar{Werke}, Begutachtung und Freigabe durch die
  \glossar{Redakteure}), für den noch ein genaues Betreiberkonzept zu
  entwickeln ist.
\end{enumerate}
Für diesen musealen Allgemeinprozess der \glossar{Dauerausstellung} steht eine
Übersicht über alle \glossar{Merkmale} zur Verfügung, mit denen
\glossar{Besucher} Ordnungs"|möglichkeiten für die \glossar{Werke} an die Hand
gegeben werden. Insofern entstehen allein durch die Such- und Filterfunktionen
verschieden strukturierte \glossar{Ausstellungen}.

\subsection{Projekte}\label{grundlagen.projekte}

\subsubsection{Projekte und Projektangebote}

In einem \glossar{Projekt} werden Lehrer typischerweise 
\begin{enumerate}
\item \emph{eigene} didaktische Zielstellungen mit den Möglichkeiten unserer
  \glossar{Plattform} umsetzen oder
\item eines unserer \glossar{Projektangebote} für die eigene Zielgruppe nutzen
  und dazu entsprechend modifizieren.
\end{enumerate}
Es ist prinzipiell möglich, auf der Basis unserer \glossar{Plattform}
\emph{freie Projekte} umzusetzen.  Details hierzu sind mit den
Plattformbetreibern in jedem Fall einzeln abzustimmen. 

Als \emph{Hauptform} orientieren wir jedoch darauf, ein Projekt auf der
Grundlage eines unserer \glossar{Projektangebote} umzusetzen. Dafür werden
unterschiedliche Lehrmaterialien in verschiedenen Szenarios als \glossar{Sets}
im Bereich ME PROJECTS zur Verfügung gestellt. Ein solches Set ist eine
Sammlung von ausführlichen Materialien (Informationen zum Thema, zum Inhalt, zu
den Lernzielen, den didaktischen Methoden, zur Zeitplanung, sowie empfohlene
Unterrichtsfächer und Altersklassen, sowie vorgefertigte Arbeitsaufträge mit
Arbeitsblättern) zur Umsetzung eines Projekts und enthält in der Regel
\begin{itemize}
\item Erklärtext mit Akzentuierung, Altersgruppe, übergreifendes Fach
\item Text zur Beschreibung
\item Grob-Übersicht (Zim-Plan) mit Ziel, Inhalt, Methode, Zeit (Tabelle, vier
  Einheiten für die Doppelstunde)
\item Stundenplanung (Tabelle) mit Ziel, Inhalt, Methode, Zeit (Tabelle mit
  Literaturhinweisen)
\item Methoden mit Glossar
\item Arbeitsblätter und Materialien (Medien).
\end{itemize}
Diese Materialien werden als komplettes Set von Materialien zum Download im
\glossar{Bereitstellungsraum} der Plattform vorgehalten.

\Kommentar{02.03.2016, Gräbe: Die Materialien sollten als Open Educational
  Resources unter einer CC-Lizenz zur Verfügung gestellt werden, welche die
  Nachnutzung \emph{und} Modifikation der Materialien erlaubt, um die
  urheberrechtliche Situation klar zu regeln.  Dem müssten dann auch die
  Autoren der Materialien zustimmen. }

Einzelne Materialien können zu mehreren \glossar{Sets} gehören, die
Zugehörigkeit zu einzelnen Projektangeboten wird durch eine
Angebotsbeschreibung im RDF-Format aufgelöst.

Das Kernteam des PVM-Projekts kümmert sich um die Weiterentwicklung dieser
Materialien, wozu insbesondere das Feedback der Projekte ausgewertet wird.
Die Erhebung dieses Feedbacks erfolgt in direktem Kontakt mit den
\glossar{Projektleitern}. Eine Unterstützung dieses Prozesses durch die
\glossar{Plattform} ist aktuell nicht vorgesehen.

\Kommentar{29.03.2016, Gräbe: Perspektivisch ist die Einbindung eines
  Fragebogen-Plugins vorgesehen.  Diese Anforderung muss noch genauer
  untersetzt und priorisiert werden. Umfragen müssen dazu konzeptionell
  genauer in den gesamten Plattformprozess eingeordnet werden.}

\subsubsection{Projekte umsetzen}

Zunächst ist vom \glossar{Projektleiter} ein \emph{Konzept} für das Projekt zu
entwickeln und einzureichen, das von einem \glossar{Redakteur} daraufhin
geprüft wird, ob es mit unserer \glossar{Plattform} in ihrem aktuellen
Ausbauzustand umgesetzt werden kann.

Ein Projekt durchläuft die Phasen \emph{Beantragung}, \emph{Durchführung},
\emph{Archiv}.  Nach Beantragung und Freigabe wird zur Durchführung des
Projekts auf der Plattform eine \glossar{Projektseite} eingerichtet, die vom
Projektleiter entsprechend der \emph{Richtlinien für Projekte} geführt und
gestaltet werden kann.

\Kommentar{02.03.2016, Gräbe: Derartige Richtlinien sind zu erstellen. }

Diese \glossar{Projektseite} ist während der Durchführungsphase als Direktlink
(öffentlich) erreichbar, aber nicht in die Menüstrukturen des PVM eingebunden.
Dies erfolgt erst nach Abschluss des Projekts und Übergabe der Seite durch den
\glossar{Projektleiter} an einen \glossar{Redakteur}, der die Einbindung der
Projektkachel in die Seite „Alle Projekte“ veranlasst.  Damit soll
gewährleistet werden, dass die Projektseite sowohl als Arbeitsseite während
der Projektdurchführung als auch als Ergebnisseite nach Projektende verwendet
werden kann und zugleich die Einheitlichkeit des Plattformdesigns
gewährleistet ist.

\Kommentar{29.03.2016, Gräbe: Es gibt noch keine Designvorgabe für die
  Struktur der Projektseite. }

\subsubsection{Standardprojekte}

Entsprechend unserem Plattformkonzept geht die \emph{zentrale Form} eines
Projekts davon aus, dass die Teilnehmer im Projektverlauf neue Werke
erstellen, wobei der \glossar{Projektleiter} in einem realen Lernsetting mit
einer Gruppe eine thematisch gebundene \glossar{Ausstellung} erstellt. Dabei
wird jeder Projektteilnehmer zum \glossar{Urheber} und erstellt und
veröffentlicht ein eigenes \glossar{Werk} zu einen gemeinsamen Thema in der
Plattform. Der \glossar{Projektleiter} fasst die Werke der Gruppe zu einer
\glossar{Ausstellung} zusammen, der gemeinsame Prozess der Werkserstellung
wird auf der Plattform auf einer \glossar{Projektseite} dokumentiert. 

Für derartige Projekte ist eine \glossar{Registrierung} auch aller Teilnehmer
an der \glossar{Plattform} zwingend erforderlich, da nur registrierte
\glossar{User} neue \glossar{Werke} hochladen können.

Vor dem Hochladen eines \glossar{Werks} müssen dessen Einzelteile am eigenen
Computer erstellt bzw. zusammengestellt werden.  Das Hochladen fertiger
\glossar{Werke} folgt dem allgemeinen Einreiche- und Freigabeprozess der
Plattform.  Es können nur solche Werke hochgeladen werden, welche die
allgemeinen Anforderungen an Werke erfüllen. Werke werden nach Durchlaufen des
standardisierten \glossar{Freigabeprozesses} der Plattform grundsätzlich
dauerhaft in den \glossar{Fundus} übernommen.  Ein Löschen von Werken aus dem
Fundus ist nur nach den allgemeinen Richtlinien der PVM-Plattform möglich.

Erst nach dieser Freigabe kann der Projektleiter ein Werk seinem
Projektkontext zuordnen und damit in die projektspezifische Ausstellung
übernehmen.

\subsection{Das PVM-Rollenkonzept}\label{grundlagen.rollenkonzept}
\begin{quote}
  \begin{tabular}{l|p{.65\textwidth}}
    \textbf{Rolle} & \textbf{Aufgaben}\\\hline
    Administrator & Konfiguration der \glossar{Plattform} sowie Verwaltung von
    Accounts und deren Rollenzuordnung\\ 
    Redakteur & Freischalten von \glossar{Werken}, Missbrauchsmeldungen
    bearbeiten\\ 
    Projektleiter & Konfiguration und Verwaltung von
    \glossar{Projekt}-Kontexten\\ 
    Urheber & Autor von eigenen \glossar{Werken}\\
    User & Registrierter und an der \glossar{Plattform} angemeldeter Nutzer\\
    Besucher & Anonyme Nutzung des öffentlichen Bereichs der
    \glossar{Plattform} 
  \end{tabular}
\end{quote}
Der öffentliche Bereich des Museums kann grundsätzlich von jedem
\glossar{Besucher} betreten werden, diese PVM-Rolle ist also auf keine WP-Rolle
abzubilden. Ein Besucher kann sich im Museum umschauen und dabei den
verschiedenen Filter- und Navigationspfaden folgen, aber selbst keine
Änderungen am Zustand des Museums vornehmen. 

Wie bei Wordpress auch stehen die Rollen zueinander in einem hierarchischen
Verhältnis, d.\,h.\ jede höherwertige Rolle verfügt auch über alle Rechte einer
niederwertigeren Rolle.  Die Rollen „Administrator“, „Redakteur“ und
„Projektleiter“ sind direkt den jeweiligen WP-Standardrollen zugeordnet und
erweitern diese um PVM-spezifische Funktionalitäten.  

\Kommentar{29.03.2016: Welcher WP-Standardrolle ist der Projektleiter
  zugeordnet? }

Die Rolle „User“ ist der WP-Rolle „Benutzer“ zugeordnet. Nur ein
\glossar{User} kann ein neues \glossar{Werk} als \glossar{Urheber} einreichen,
allerdings kann es Urheber geben, die nicht mehr als User an der Plattform
registriert sind.  Ein Urheber, der zugleich User ist, hat eine eigene
\glossar{Profilseite}. Zu einem nicht mehr an der Plattform aktiven Urheber
wird ein standardisiertes personenspezifisches FOAF-Profil vorgehalten, aber
keine Zuordnung zu einer WP-Rolle vorgenommen.

\Kommentar{29.03.2016: Es ist ein weiterer Menüpunkt „Inspiration“
  hinzugekommen, der als Blog von Usern in der Rolle \glossar{Blogautor}
  geführt werden soll. Die Zulassung als Blogautor soll an das Ausfüllen des
  Plattform-Fragebogens gekoppelt werden und wird durch einen \glossar{Admin}
  veranlasst. }

\subsection{Prozess der Werkerstellung}\label{grundlagen.werkerstellung}

In der \glossar{Werkstatt} werden \glossar{Werke} neu erstellt. Der
\glossar{Werkstattprozess} wird (perspektivisch) durch ein Autorenwerkzeug
unterstützt.  Der Werkstattprozess schließt mit dem \glossar{Einreichen} des
fertigen Werks im \glossar{PVM-Paketformat}.

Im Release~1 der Plattform wird das \glossar{Einreichen} eines fertigen Werks
nur durch eine \glossar{Formularseite} unterstützt, über welche die Einzelteile
eines \glossar{Pakets} zusammengetragen und valide im
\glossar{PVM-Paket"|format} gespeichert werden können.

Das Werk erscheint nicht direkt im Frontend der \glossar{Plattform}, also in
der \glossar{Dauerausstellung}, sondern muss dafür durch einen
\glossar{Redakteur} \glossar{freigegeben} werden.

Ein Werk kann einem oder mehreren \glossar{Projekten} zugeordnet werden.

\subsection{Entwicklungs- und Betreiberkonzept}
\label{grundlagen.betreiberkonzept}

Die \glossar{Plattform} entsteht auf der Basis einer
Standard-Wordpress-Installation mit ausgewählten unveränderten Plugins, über
die abgegrenzte Funktionalitäten realisiert werden (etwa die Registrierung und
Verwaltung von \glossar{Usern}) sowie eigenen \textbf{PVM-Plugins}, in denen
die plattformspezifischen Funktionalitäten gebündelt werden.  Das Design der
Plattform wird über ein eigenes \textbf{PVM-Theme} realisiert, das als Child
Theme das Basisthemes
\emph{Esteeme}\footnote{\url{https://wordpress.org/themes/esteem/}} realisiert
wird.  Bei der Abgrenzung der Zuordnung einzelner Funktionalitäten zu den
PVM-Plugins oder zum PVM-Theme ist auf eine klaren Trennung zwischen
Funktionalität und Layout/Design zu achten.

Bei der Auswahl des Basisthemes wurde darauf geachtet, dass es \emph{responsive
  Design} unterstützt.

Es soll möglich sein, die Plattform auf einem anderen Server auszurollen.  Dazu
ist ein entsprechendes \textbf{Relokationskonzept} zu entwickeln.

Dies ist nur durch ein iteratives Entwicklungsmodell zu erreichen, in dem
konzeptionelle, inhaltliche und technische Weiterentwicklung der Plattform
angemessen miteinander verzahnt werden.  Technisch ist dazu zwischen der
Produktiv- und einer Entwicklungsplattform zu unterscheiden sowie ein stimmiges
Release-Konzept zu entwickeln, nach dem die Produktivplattform periodisch
aktualisiert wird.

\Kommentar{4.12.2015, Düwel: Die Site bekommt im April 2015 eine Domain von
  Strato, Kosten trägt das Institut für Kunstpädagogik. Sie liegt auf dem
  Server der Uni (Informatik), braucht aber einen kurzen Namen und muss
  jederzeit exportierbar sein, es muss gewährleistet sein, dass sie immer
  läuft.}

\Kommentar{5.12.2015, Gräbe: Für die Gewährleistung des Betriebs müssen die
  erforderlichen personellen Ressourcen bereitgestellt werden.}

\subsubsection{Rollen im Betriebskonzept} 

Zum Betrieb der Wordpress-Produktivinstanz der PVM-Plattform sind folgende
Rollen zu besetzen:
\begin{itemize}
\item \emph{Site-Admin} --- zuständig für die Installation und Aktualisierung
  der grundlegenden Software auf dem Rechner, das Basis-Backup sowie Kontakt
  bei Rechnerausfällen und Neustart.
\item \emph{Portal-Admin} --- Aktualisierung der Wordpress-Instanz auf eine
  neue Version, Installation und Aktualisierung von Plugins und des
  PVM-eigenen Themas, Upload-Verwaltung.  Ein Portal-Admin hat Zugriff auf das
  Installationsverzeichnis auf dem Server.
\item \emph{WP-Admin} --- zuständig für konzeptionelle Fragen sowie die
  Konfiguration der verschiedenen Anwendungen, Plugins und Funktionen über die
  Admin-Konfigurations-Oberfläche von Wordpress. Ein WP-Admin hat keinen
  Zugriff auf das Installationsverzeichnis auf dem Server.
\item \emph{WP-Redakteure} und \emph{WP-Autoren} --- zuständig für die Inhalte
  der Plattform.
\end{itemize}

\section{Anwenderszenarien}

\subsection{Möglichkeiten für nicht angemeldete Besucher} 

\subsubsection{Werke suchen und zusammenstellen}

Die Such- und Filterfunktion verfügt über zwei grundsätzliche Funktionen. Dies
ist zum einem die Volltextsuche, die bei allen \glossar{Werken} im Titel und im
dazugehörigen Text nach Übereinstimmungen zur Suchphrase sucht und sich somit
fast nicht von der Wordpress-Standard-Suche unterscheidet. Zum anderen gibt es
eine Suche nach \glossar{Merkmalen}, welche nur diejenigen \glossar{Werke}
zurückgibt, welche mit der ausgewählten Merkmalskombination markiert sind.

Bei beiden Suchfunktionen kann man die gefilterten Werke nach verschiedenen
Kriterien (etwa Beliebtheit oder Erstellungsdatum) sortieren lassen.

\subsection{Möglichkeiten für \glossar{User}}

\subsubsection{Registrierung und Authentifizierung}

Für eine Registrierung sind die folgenden Angaben obligatorisch:
\begin{itemize}
\item obligatorisch: Name, Benutzername (nick name), Email, Standort 
\item optional: eigenes Foto, Verweis auf eigene Webpräsenz 
\end{itemize}
Der Standort kann aus einer liste vorgegebener Standorte ausgewählt
werden. Die Auswahlliste der Standorte liegt im RDF-Format vor und kann durch
einen Portal-Admin erweitert werden. 

Im Zuge der Registrierung als \glossar{User} sind die AGB zu bestätigen.
Während des Einreichens von \glossar{Werken} wird keine weitere AGB-Bestätigung
abgefragt; die AGB müssen also auch alle rechtlichen Fragen behandeln, die mit
dem Einreichen von Werken durch einen \glossar{Urheber} zu klären sind.

\subsubsection{Benutzerprofil und \glossar{Profilseite}}

Ein \glossar{User} kann nach Anmeldung im WP-Backend auf sein Benutzerprofil
zugreifen und dort alle bei der Registrierung angegebenen Daten einsehen sowie
diese bei Bedarf, mit Ausnahme des Nutzernamen, ändern.  Weiterhin hat er dort
die Möglichkeit, eine kleine Biographie anzugeben und sein Passwort sowie die
Ortsangabe aus einer vordefinierten Auswahl zu ändern. 

Aus diesen Daten wird die \glossar{Profilseite} des Users erzeugt.  Auf der
Profilseite sind neben den Angaben zum \glossar{User} dessen \glossar{Werke}
sowie dessen \glossar{Bewertungen} der Werke anderer gelistet.

Name, Ortsangabe und Profilbild bilden als \texttt{foaf:name},
\texttt{foaf:based\_near} und \texttt{foaf:Image} die Basis für ein
standardisiertes FOAF-Profil des Users, das auch nach Löschen des Accounts in
der Plattform erhalten bleibt, um Urheberschaften referenzieren zu können.

\subsubsection{Werke bewerten}

Ein \glossar{User} findet auf jeder Einzelseite eines Werks eine Möglichkeit,
für dieses Werk eine Bewertung aus einer global vorgegebenen Liste von
Bewertungen abzugeben oder zu modifizieren. Jeder User kann jedes Werk nur
einmal bewerten, diese Bewertung aber ändern. 

\subsubsection{Ein neues Werk zusammenstellen und einreichen}

Ein \glossar{User} kann ein neues \glossar{Werk} als valides \glossar{Paket} im
\glossar{PVM-Paketformat} zusammenstellen, in den \glossar{Bereitstellungsraum}
hochladen und zur \glossar{Freigabe} anmelden.  Dieser \glossar{User} ist damit
zugleich der \glossar{Urheber} dieses Werks.

Dazu erstellt ein \glossar{Urheber} zunächst sein \glossar{Werk} entsprechend
den Vorgaben.  Ort dieses Prozesses ist die \glossar{Werkstatt}, vorbereitende
Arbeiten kann der Urheber auf seinem eigenen PC ausführen.  

Diese Phase schließt mit der \glossar{Einreichung} eines validen
\glossar{Pakets} ab.  Dabei wird der \glossar{Urheber} durch die Bereitstellung
einer \glossar{Formularseite} unterstützt, über welche die einzelnen
\glossar{Werkteile} ausgewählt, zu einem validen \glossar{Paket} im
\glossar{PVM-Paketformat} zusammengefasst und in den
\glossar{Bereitstellungsraum} der \glossar{Plattform} hochgeladen werden
können.

Danach beginnt der Prozess der \glossar{Freigabe}.

\subsubsection{Freigabe eines Werks}

Der \glossar{Freigabe}-Prozess wird mit Abschluss der \glossar{Einreichung}
eines \glossar{Werks} ge"|startet.  In dieser Zeit befindet sich das Werk in
einer Art Zwischenstadium, in dem der \glossar{Urheber} keine Änderungen mehr
am Werk vornehmen kann, das Werk aber auch noch nicht in den \glossar{Fundus}
übernommen ist.

Die \glossar{Freigabe} erfolgt durch einen \glossar{Redakteur}, der das Werk
nach entsprechender Prüfung für den \glossar{Fundus} freigibt oder aber das
Werk zurückweist. Die Freigabe durch einen Redakteur soll in erster Linie das
Einhalten der AGB's (die jeder \glossar{User} beim Erstellen seines Accounts
bestätigt) sowie eine saubere Zuordnung gewährleisten.

\Kommentar{15.01.2016, Gräbe: Der Redakteur achtet dabei vor allem auf formale
  Gesichtspunkte. Hierzu ist eine genauere Auf"|listung der zu beachtenden
  Aspekte als Handreichung für den \glossar{Redakteur} zu entwickeln.}

\subsubsection{Mitarbeit in einem Projekt}

\glossar{User} können auf Einladung durch den \glossar{Projektleiter} in einem
\glossar{Projekt} mitarbeiten.  Die Mitarbeit im Projekt orientiert sich an den
\emph{projektspezifischen Zielen} und erfolgt unter Anleitung und Koordinierung
des Projektleiters.

Im Rahmen einer solchen Mitarbeit ist es möglich, eigene \glossar{Werke} aus
dem \glossar{Fundu}s in den Projektkontext über einen
\glossar{Projektfreigabeprozess} einzubringen (und auch wieder aus dem Projekt
zu entfernen).

Werke können mehreren \glossar{Projekten} zugeordnet sein. 

\subsubsection{Löschen eines Werks}\label{werk.loeschen}

Das Löschen eines Werks sollte der Ausnahmefall bleiben, kann aber aus
verschiedenen Gründen erforderlich sein:
\begin{itemize}
\item Externe Gründe: Die Auf"|forderung zum Entfernen eines Werks wird von
  außen an die Plattformbetreiber herangetragen.
\item Interne Gründe: Der \glossar{Urheber} fordert die Plattformbetreiber
  auf, sein Werk oder seine Werke zu entfernen.
\end{itemize}
Die Auf"|forderung zum Löschen eines Werks ist grundsätzlich an einen
\glossar{Admin} zu richten, der das Weitere veranlasst.

Vor dem Löschen des Werks prüft der Admin, welche \glossar{Projekte} von der
Löschung betroffen sind, welche Auswirkung das Löschen auf diese Projekte hat,
ob diese Projekte ggf. selbst zu löschen sind oder modifiziert werden müssen,
löst die entsprechenden Aufträge an die \glossar{Projektleiter} aus und
überwacht den Fortgang.

Beim Löschen eines Werks werden die Verweise auf dieses Werk aus allen
\glossar{Projekten} gelöscht, sämtliche werkrelevanten Dateien aus der
\glossar{Plattform} und dem \glossar{Bereitstellungsraum} entfernt und danach
der Status des Werks auf „gelöscht“ gesetzt.  Es verbleibt eine rudimentäre
standardisierte Information zum Werk im System.

\subsubsection{Löschen eines Users}\label{nutzer.loeschen}

Ein \glossar{User} kann bei einem \glossar{Admin} beantragen, dass sein Account
gelöscht wird.  Von einem gelöschten User wird der \texttt{foaf:name} und eine
Orts-URI als \texttt{foaf:based\_near} als standardisiertes FOAF-Profil zu
Referenzzwecken gespeichert, die \glossar{Profilseite} wird gelöscht.

Beim Löschen eines Useraccounts verbleiben die \glossar{Werke} des Users in
der Regel im \glossar{Fundus} und können entsprechend den Nutzungsrechten auch
weiterhin in der \glossar{Plattform} verwendet werden.  Die Urheberschaft an
diesen Werken wird über das verbliebene standardisierte FOAF-Profil
ausgewiesen, der Verweis auf die Profilseite durch einen standardisierten
Verweis ersetzt, dass der \glossar{Urheber} dieses Werks auf der Plattform
nicht mehr als \glossar{User} aktiv ist.

Der User kann in Ausnahmefällen zusätzlich beim Admin beantragen, dass mit dem
Löschen des Accounts auch alle seine Werke zu löschen sind. In diesem Fall hat
der Admin zusätzliche Aktivitäten nach Abschnitt \ref{werk.loeschen} in die
Wege zu leiten.  

Wenn zum User keine Werke mehr in der Plattform vorhanden sind, kann auch das
verbliebene standardisierte FOAF-Profil aus der Plattform gelöscht werden.

\subsection{Möglichkeiten für \glossar{Projektleiter}}

\subsubsection{Allgemeines}
 
Projekte können nur von \glossar{Projektleitern} erstellt werden, wozu ein
registrierter \glossar{User} von einem \glossar{Administrator} durch Anpassen
der zugeordneten Rolle autorisiert werden muss.

Projektleiter erstellen und verwalten \glossar{Projekte} im Rahmen ihrer
Möglichkeiten und Intentionen. Wenn der Projektleiter sein Projekt löscht, so
wird bei allen \glossar{Werken}, die bisher diesem Projekt zugeordnet waren,
lediglich die Projektzugehörigkeit gelöscht, nicht jedoch die Werke selbst.

\subsubsection{Projekt erstellen und bestätigen}

\begin{itemize}
\item Ein neues \glossar{Projekt} kann nur von einem \glossar{Projektleiter}
  erstellt werden.
\item Der Projektleiter reicht über die \emph{Projektformularseite} einen
  \emph{Antrag} auf ein neues Projekt ein, welcher (wenigstens) Projektkürzel,
  Titel, Projektfoto und \emph{Projektkonzept} umfasst. Aus den Metadaten
  werden Projektleiter und Einreichungsdatum ergänzt.

  Der Bezug zu \glossar{Projektangeboten} ist nur lose, neue Projekte müssen
  sich nicht auf eines der Projektangebote beziehen.

\item Ein solcher \emph{Projektantrag} wird von einem \glossar{Redakteur}
  begutachtet und bestätigt oder abgelehnt.  Die Entscheidung wird dem
  Einreicher per E-Mail mitgeteilt. 

\item Für einen bestätigten \emph{Projektantrag} legt der \glossar{Redakteur}
  die \glossar{Projektseite} an.  In der Durchführungsphase ist diese Seite
  nur über einen Direktlink zu erreichen.

  Die Seite wird erst nach Abschluss des Projekts als Kachel auf der Seite
  „Alle Projekte“ integriert und damit in die Menüstrukturen übernommen.

\item Nach dem Anlegen dieser Projektinfrastruktur in der Plattform wird der
  Projektleiter informiert und kann sein Projekt im Rahmen der
  \emph{Richtlinien für Projekte} umsetzen. 

\Kommentar{02.03.2016, Gräbe: Derartige Richtlinien sind zu erstellen. }

\end{itemize}

\subsubsection{Zuordnen von Teilnehmern zu einem Projekt}

\begin{itemize}
\item Ein \glossar{User} wird \glossar{Projektleiter}, indem er dies bei einem
  \glossar{Admin} beantragt und von diesem der WP-Rolle Projektleiter
  zugeordnet wird.  Ein User kann erst dann einen Projekterstellungsprozess
  starten, wenn er als Projektleiter registriert ist.

\item \emph{Projektteilnehmer} registrieren sich zunächst als \glossar{User}
  und werden dann vom \glossar{Projektleiter} in das Projekt eingeladen.  Nach
  der Bestätigung der Teilnahme durch den User ist dieser dem Projekt
  zugeordnet und dessen Werke im Projektkontext sichtbar.  
\item Der Projektleiter kann Teilnehmer wieder aus dem Projekt austragen.  Der
  Teilnehmer bleibt dabei \glossar{User}.  Das Löschen als User ist ggf. in
  einem weiteren Schritt zu beantragen, siehe Abschnitt \ref{nutzer.loeschen}.  
\end{itemize}

\subsubsection{Zuordnen von Werken zu einem Projekt}

\begin{itemize}
\item Einem \glossar{Projekt} können nur \glossar{Werke} zugeordnet werden,
  die bereits in den \glossar{Fundus} aufgenommen wurden und damit den
  allgemeinen Prozess der \glossar{Freischaltung} von Werken durchlaufen
  haben.  Dies gilt auch für Werke von \emph{Projektteilnehmern}.

\item Der \glossar{Projektleiter} kann seinem Projekt in der Durchführungsphase
  Werke von Projektteilnehmern zuordnen und solche Zuordnungen auch wieder
  entfernen.  Dazu wird ihm im WP-Backend eine Liste der Werke aller
  Projektteilnehmer angezeigt, die einzeln für das Projekt an- und abgewählt
  werden können.  Eine Zustimmung des jeweiligen Projektteilnehmers ist dabei
  nicht erforderlich.  Allerdings wird davon ausgegangen, dass diese Zustimmung
  im begleitenden pädagogischen Prozess eingeholt wurde.

  Zuordnen und Entfernen eines Werks ändern dessen Zugehörigkeit zum
  \glossar{Fundus} nicht.

\item Ein Werk kann gleichzeitig in mehreren Projektkontexten referenziert
  sein.

\end{itemize}

\subsubsection{Löschen von Projekten}

Das Löschen eines \glossar{Projekts} sollte nur in Ausnahmefällen geschehen
und ist durch den \glossar{Projektleiter} bei einem \glossar{Admin} zu
beantragen.

Beim Löschen des Projekts wird die \glossar{Projektseite} aus dem PVM entfernt
und alle Verbindungen von Teilnehmern und \glossar{Werken} zu diesem Projekt
gelöscht.  Dies hat weder Einfluss auf die Zugehörigkeit eines \glossar{Werks}
zum \glossar{Fundus} noch auf die Registrierung eines Teilnehmers des Projekts
als \glossar{User}.

Über das Löschen des Projekts hinausgehende Aktionen (Löschen von
\glossar{Usern} oder \glossar{Werken}) sind einzeln wie in den Abschnitten
\ref{werk.loeschen} und \ref{nutzer.loeschen} beschrieben zu beantragen und zu
prozessieren.

\subsection{Möglichkeiten für \glossar{Redakteure}}

Aufgabe der Redakteure ist die \glossar{Freigabe} eingereichter \glossar{Werke}
sowie die Bearbeitung von gemeldeten Verstößen.  Weiterhin können sich
Redakteure über statistische Informationen zur Nutzung der \glossar{Plattform}
informieren. 

\paragraph{Freigabe eingereichter Werke:} 
Dem Redakteur wird dazu eine Liste aller zur Freigabe anstehenden
\glossar{Werke} im \glossar{Bereitstellungsraum} angezeigt.  Zur Prüfung eines
dieser Werke kann der Redakteur das Werk in einem geschützten Modus zeitweise
im \glossar{Fundus} verfügbar machen und später freischalten oder wieder aus
dem Fundus entfernen.  Der \glossar{Redakteur} informiert den \glossar{Urheber}
über das Ergebnis der Prüfung.

\Kommentar{15.01.2016, Gräbe: Hat der Urheber bei Ablehnung die Möglichkeit zum
  Einspruch und wie wird dieser behandelt? Wie wird mit \glossar{Werken}
  umgegangen, die im \glossar{Bereitstellungsraum}, nicht aber im
  \glossar{Fundus} sind? Es ist ein Workflow zu definieren, unter welchen
  Umständen und wie Werke wieder aus dem \glossar{Bereitstellungsraum} entfernt
  werden dürfen.}

\paragraph{Gemeldete Werke:} 
\glossar{User} sehen zu jedem \glossar{Werk} auf dessen \emph{Werkseite}
einen Link „Verstoß melden“.  Dieser dient dazu, die \glossar{Redakteure} auf
bereits veröffentlichte Werke aufmerksam zu machen, die aus verschiedenen
Gründen den Standards des PVM nicht gerecht werden.  Es kann zum Beispiel
vorkommen, dass sich ein User absichtlich oder unwissentlich nicht an die AGB
hält oder Urheber-Rechte verletzt wurden. Um dagegen vorgehen zu können
existiert die Melde-Funktion.

Über diesen Link wird der \glossar{User} zu einem Formular weitergeleitet, über
das er an die \glossar{Redakteure} innerhalb der Website eine Nachricht senden
kann. Nach dem Abschicken dieser Nachricht wird im Backend unter „Gemeldete
Werke“ ein neuer Eintrag angelegt, der nur von den Redakteuren eingesehen und
bearbeitet werden kann. Ein Redakteur kann das entsprechende \glossar{Werk}
entweder wieder „entwarnen“ (also die Meldung löschen) oder das Werk löschen,
siehe Abschnitt \ref{werk.loeschen}.

Gemeldete Werke bleiben sichtbar, bis sie ggf. gelöscht werden, werden also
nicht präventiv aus der \glossar{Dauerausstellung} entfernt.

\paragraph{Statistik:} 
Redakteure können Statistik-Informationen über den Besuch einzelner Seiten der
Plattform auswerten.  Hierzu wird ein WP-Standard-Plugin installiert, das im
Backend ein Statistik-Menü zur Verfügung stellt.  Innerhalb des Statistik-Menüs
können keine Einstellungen vorgenommen, sondern nur Informationen ausgelesen
werden.

\section{Konzeptionelle und Design-Fragen}

\subsection{Design}

Mit einer schlichten Oberfläche zur Darstellung der \glossar{Werke} soll
erreicht werden, dass die Umgebung sauber und aufgeräumt wirkt, die einzelnen
Werke in ihrer Ganzheit wirken und nichts von ihrer direkten Rezeption ablenkt.
So gibt es zum Beispiel keine Animationen oder starke und unpassende
Farbkontraste.

Der Einstieg in die \glossar{Plattform} erfolgt über die \emph{Startseite}.
Weiter gibt es 
\begin{itemize}
\item eine \emph{Seite „Alle Werke“} mit Such- und Filterfunktion, auf der alle
  \glossar{Werke} als Kacheln angezeigt werden,
\item für jeden \glossar{User} eine \emph{Profilseite}, 
\item für jedes \glossar{Werk} eine \emph{Werkseite},
\item eine \emph{Seite „Alle Projekte“}, auf der alle (abgeschlossenen)
  \glossar{Projekte} als Kacheln angezeigt werden, 
\item für jedes \glossar{Projekt} eine \emph{Projektseite}
\item eine \emph{Seite „Alle Projektangebote“}, auf der alle
  \glossar{Projektangebote} als Kacheln angezeigt werden, sowie
\item für jedes \glossar{Projektangebot} eine \emph{Projektangebotsseite}. 
\end{itemize}
Um die Einheitlichkeit der Darstellung zu sichern wird für jede dieser Seiten
bzw.\ Seitentypen im PVM-Theme jeweils ein eigenes \emph{Seitentemplate}
angelegt.  

Daneben gibt es weitere Seiten mit erklärendem Charakter -- „About“,
„Tutorium“, „Impressum“ -- sowie die WP-Standardfunktionen \emph{Login},
\emph{Registrierung} und \emph{Passwort-Reset}, die in die Menüführung
eingeordnet sind.

\Kommentar{29.03.2016: Es ist ein weiterer Menüpunkt „Inspiration“
  hinzugekommen, der als Blog von Usern in der Rolle \glossar{Blogautor}
  geführt werden soll. }

Die Nutzerführung erfolgt über das Menü sowie über die Seiten „Alle Werke“ und
„Alle Projekte“, auf denen die einzelnen Items als Bildkacheln dargestellt
sind. Über Filterfunktionen kann die Anzahl der angezeigten Items reduziert
werden.  Damit wird ein übermäßiges Aufblähen des Menüs verhindert.

Einzelne Seiten sind auch über Direktlinks zu erreichen.  Die Projektseiten
noch nicht abgeschlossener Projekte sind \emph{nur} über ihren Direktlink zu
erreichen.

\subsection{Frontend und Backend}

Wordpress unterscheidet zwischen dem Präsentationsbereich (Frontend) und dem
Administrationsbereich (Backend).  Während das Frontend das „Schaufenster“ der
Anwendung ist, kann das Backend grundsätzlich nur von angemeldeten Benutzern
per Navigation über die nach der erfolgreichen Anmeldung am oberen Rand
sichtbare schwarze WP-Standard-Leiste erreicht werden.  Das „Look and Feel“ des
Frontends ist in dem Umfang einheitlich wie dies das Design der Plattform
vorsieht, das Backend sellt rollenabhängig Funktionalität in verschiedenem
Umfang bereit.

Das Backend wird ggf. für die Rollen \glossar{Redakteur} und
\glossar{Projektleiter} um PVM-spezifische Funktionen erweitert.  User in
diesen Rollen müssen also hinreichend mit dem allgemeinen „Look and Feel“ des
WP-Backends vertraut sein.

Das User-Profil kann ebenfalls über das WP-Backend editiert werden, wobei hier
nur die Standardmöglichkeiten des Plugins \texttt{wp-members} genutzt werden,
die durch einen Administrator plattformspezifisch konfiguriert werden können.
Das ist bei der Abfassung einer Handreichung ggf. zu berücksichtigen. 

\subsection{Seitenaufbau der Plattform}

Details zum Seitenaufbau der Plattform werden nur in dem Umfang in den
Requirements erfasst, in dem sich der Seitenaufbau \textbf{nicht} allein durch
Konfiguration der Plattform im Administratoren-Menü des ausgewählten
Basisthemes gestalten lässt, sondern vorbereitende Arbeiten im PVM-Theme oder
im PVM-Plugin erforderlich sind.  Für jeden der folgenden Seitentypen ist ein
eigenes Seitentemplate zu entwickeln. 

\subsubsection{Startseite}

Die \emph{Startseite} ist die Einstiegsseite in die Plattform.  Dort sollen
einerseits die verfügbaren Merkmale und andererseits die im Fundus vorhandenen
Werke aufgelistet bzw. präsentiert werden.  

\subsubsection{Übersichtsseite „Alle Projekte“}
Auf der \emph{Übersichtsseite „Alle Projekte“} werden alle abgeschlossenen
\glossar{Projekte} gelistet.  Jedes abgeschlossene Projekt wird durch eine
Kachel repräsentiert.

\subsubsection{Übersichtsseite „Alle Werke}
Auf der \emph{Übersichtsseite „Alle Werke“} werden alle vorhandenen
\glossar{Werke} gelistet und verschiedene Such- und Filterfunktionen
angeboten, um die Zahl der angezeigten \glossar{Werke} zu beschränken.  Jedes
Werk wird durch eine Kachel repräsentiert.

\subsubsection{Profilseite eines Users}
Jeder an der \glossar{Plattform} registrierte Nutzer (\glossar{User}) hat eine
Profilseite, auf der im Zuge des Anmeldeprozesses erhobene Pflichtinformationen
zu seiner Person angezeigt werden und die er durch persönliche Informationen im
festzulegenden Umfang ergänzen kann.  Auf der Profilseite werden die
\glossar{Werke} des Users gelistet und dessen \glossar{Bewertungen} anderer
\glossar{Werke}.  

Der User kann die Profilseite nicht direkt editieren. Es gibt ein
Profilseiten-Template mit einer einzigen (leeren) Profilseite, der User wird
über die Autor-ID als Get-Parameter in der URL übergeben und darüber die
hinterlegte User-Information ausgelesen. 

\subsubsection{Projektseite}
Seite, auf der ein einzelnes \glossar{Projekt} angezeigt wird.  Diese Seite
wird erst auf der Seite „Alle Projekte“ verlinkt, wenn das Projekt
abgeschlossen ist. 

\subsubsection{Werkseite}
Seite, auf der ein einzelnes \glossar{Werk} angezeigt wird.  Sidebar zeigt
dann die \glossar{Merkmale} des Werks und die \glossar{Bewertungen} durch
\glossar{User}.  Im eingeloggten Zustand wird dem User die Möglichkeit
angeboten, eine eigene Bewertung abzugeben oder zu ändern.

\newpage
\section{Glossar} 

\paragraph{Administrator:} 
Standard-WP-Rolle, die auf alle Daten und Funktionen innerhalb des Systems
zugreifen kann, Nutzeranmeldungen freischaltet, Nutzern Rollen zuweist und die
Plattform konfiguriert.  Sie ist allen anderen Rollen in diesem Sinne
übergeordnet.

\paragraph{Ausstellung:} 
Eine Zusammenstellung und Präsentation von \glossar{Paketen} aus dem
\glossar{Fundus} in einem spezifischen didaktischen {Kontext}.

Das PVM ist als Virtuelles Museum insbesondere eine \emph{virtuelle
  Ausstellung}, die sich von einer \emph{digitalisierten Ausstellung} insoweit
unterscheidet, als letztere eine \emph{reale} Ausstellung digital abbildet.
Wesentlicher Unterschied ist dabei, dass in virtuellen Ausstellungen ein
Ausstellungsobjekt \emph{gleichzeitig} in \emph{mehreren} Ausstellungen
gezeigt werden kann.

\paragraph{Bereitstellungsraum:} 
Bereich der \glossar{Plattform}, in welchem alle (sowohl die freigegebenen als
auch die noch auf ihre Freigabe wartenden) \glossar{Werke} im
\glossar{PVM-Paketformat} abgelegt sind sowie weitere global verfügbare
Materialien vorgehalten werden.

\paragraph{Besucher:} 
Als Besucher wird eine nicht eingeloggte Person bezeichnet, die den
öffentlichen Bereich des PVM nutzt.

\paragraph{Bewertungssystem:} 
System fest vorgegebener Begriffe mit bildlichen Icons, das \glossar{User} zur
Bewertung von \glossar{Werken} einsetzen können.

Die Überwachung freier Kommentare ist mit den verfügbaren Personalressourcen
nicht zu bewältigen, deshalb wird es (wenigstens in dieser Ausbaustufe des PVM)
keine Möglichkeit zu freien Kommentaren und auch kein Forum geben.  

\paragraph{Bild-Teil:}
Spezielle Art von \glossar{Werkteil}, das eine Bilddatei im png-Format enthält.
Um ein einheitliches Erscheinungsbild innerhalb der \glossar{Plattform} zu
garantieren, muss ein Bild-Teil spezielle \emph{Designvorgaben} erfüllen.

\paragraph{Blogautor:} 
Ausgewählte \glossar{User} in der WP-Rolle „Autor“, die den Blog unter dem
Menüpunkt „Inspiration“ führen. 

\paragraph{Dauerausstellung:} 
Präsentation aller \glossar{Pakete} aus dem \glossar{Fundus} im (näher zu
definierenden) \emph{Allgemeinkontext} des PVM-Projekts.

\paragraph{Einreichung:} 
Ein \glossar{Urheber} lädt ein \glossar{Werk} als valides \glossar{Paket} im
\glossar{PVM-Paket"|format} über die Eingabeschnittstelle in den
\glossar{Bereitschaftsraum} der \glossar{Plattform} hoch.

\paragraph{FOAF:} 
Friends of a Friend, eine RDF-Ontologie, siehe
\url{http://xmlns.com/foaf/0.1/}.

\paragraph{Formularseite:} 
Webseite, über welche die Einzelteile eines \glossar{Pakets} zusammengetragen
und valide im \glossar{PVM-Paketformat} gespeichert werden kön"|nen.  

Damit erfährt im Release~1 des PVM der \glossar{Werkstattprozess} rudimentäre
Unterstützung.

\paragraph{Freigabe:} 
Prüfprozess, der nach der \glossar{Einreichung} eines \glossar{Werks} durch den
\glossar{Urheber} einsetzt. Ein \glossar{Redakteur} entscheidet über Freigabe
oder Zurückweisung des eingereichten {Werks}.

\paragraph{Fundus:} 
Sammlung aller aktuell veröffentlichten \glossar{Werke}.  Alle
veröffentlichten Werke sind auch im \glossar{PVM-Paketformat} im
\glossar{Bereitstellungsraum} abgelegt.

\paragraph{Merkmal:} 
Paare (Merkmalsart, Ausprägung), die in einem \glossar{Paket} vom
\glossar{Urheber} zur Charakterisierung von Paketen verwendet werden können.
Die möglichen Merkmale sind als \emph{morphologische Tabelle} in Form einer
\glossar{Merkmalsontologie} vorgegeben.

\paragraph{Merkmalsontologie:} 
Im Rahmen des Systems fest vorgegebene \emph{morphologische Tabelle}, die als
\glossar{SKOS}-basierten Ontologie vorliegt, in der Ausprägungen als
\texttt{skos:Concept} und Parameter (Merkmalsklassen) -- Farbe, optische
Eigenschaften, Material, Oberfläche, Größe, Alter -- als
\texttt{skos:ConceptScheme} konzeptualisiert werden.

Die Darstellung des Parameters „Standort“ ist noch gesondert zu diskutieren, da
er konzeptionell anderer Art ist (Wertebereich ist nicht fixiert).

\Kommentar{5.11.2015, Gräbe: Die morphologische Tabelle wurde in der Zuarbeit
  Düwel vom 29.09.2015 vorgegeben und in die Form einer
  \glossar{SKOS}-Ontologie gebracht, die in der Datei \texttt{Merkmale.ttl} im
  RDF-Format vorliegt. }

\paragraph{Paket:} 
Grundlegende Einheit einer \glossar{Ausstellung} oder eines \glossar{Projekts},
das
\begin{itemize}\itemsep0pt
\item ein Container verschiedener \glossar{Werkteile} ist,
\item einer festen Struktur folgt, die als \glossar{Paket-Metainformation} in
  digitaler Form Teil des Pakets ist,  
\item von einem \glossar{Urheber} im Zuge eines \glossar{Werkstattprozesses}
  in einem privaten Raum geschaffen und zur Veröffentlichung vorgeschlagen
  wurde und
\item von einem \glossar{Redakteur} zur Veröffentlichung freigegeben wird.
\end{itemize}
Ein Paket kann nach der \glossar{Freigabemeldung} nicht mehr verändert werden.
Physisch liegen veröffentlichte Pakete als Digitalisate in Dateiform im
\glossar{PVM-Paketformat} im \glossar{Bereitstellungsraum}.

\paragraph{Paket-Metainformation:}
Digital lesbare Beschreibung des Inhalts eines \glossar{Pakets}, die einer
\emph{Beschreibungsspezifikation} folgt.  Um maximale Kompatibilität zu
\emph{Linked Open Data}\footnote{\url{http://linkeddata.org/}.} Konzepten zu
gewährleisten, werden Beschreibung und Beschreibungsspezifikation über RDF als
Framework realisiert.

\paragraph{Plattform:} 
Synonym für die zunächst unter der Bezeichnung \glossar{Museum} gefasste
Erweiterung des Begriffs \glossar{Ausstellung} um eine
didaktisch-konzeptionelle Dimension. 

\paragraph{Profilseite:} 
Jeder \glossar{User} hat eine eigene solche Profilseite.  

\paragraph{Projekt:} 
In einem \glossar{Projekt} können Lehrer \emph{eigene} didaktische
Zielstellungen mit den Möglichkeiten unserer \glossar{Plattform} umsetzen oder
eines unserer \glossar{Projektangebote} für die eigene Zielgruppe nutzen und
dazu entsprechend modifizieren.

\paragraph{Projektangebot:} 
Spezifischer didaktischer Rahmen zur Nutzung der Plattform. Im Rahmen des PVM
werden Projektangebote systematisch entwickelt und dazu \glossar{Sets} aus
Materialien bereitgestellt.

\paragraph{Projektfreigabeprozess:} 
Prozess, nach dem ein \glossar{Werk} in ein \glossar{Projekt} eingebracht
wird. Werke können nur von Projektteilnehmern eingebracht werden. Die
Einbringung erfordert die Zustimmung des \glossar{Projektleiters}. Werke können
ausschließlich vom Projektleiter wieder aus dem Projekt entfernt werden,
verbleiben dabei aber im \glossar{Fundus}.

\paragraph{Projektleiter:} 
Eine Erweiterung der WP-Rolle „Mitarbeiter“ um das Recht, Projekte anzulegen
und zu verwalten.  Jedes \glossar{Projekt} verfügt über einen Projektleiter.

\paragraph{PVM-Paketformat:}
Spezifischen Binding der \glossar{PVM-Paketstruktur}, um ein \glossar{Paket}
als Digitalisat in einer einzelnen Datei zu speichern.

\paragraph{PVM-Paketstruktur:}
Innere Struktur eines Pakets als in RDF beschriebenes \emph{Datenmodell} samt
eines spezifischen Bindings an das \glossar{PVM-Paketformat}.  Ein Paket
enthält
\begin{itemize}
\item Urheber, Entstehungszeitpunkt, Ort des Urhebers, Titel als
  \glossar{Paket-Metainforma"|tion}, 
\item ein vom \glossar{Urheber} selbst erstelltes oder urheberrechtlich
  zulässig nachgenutztes \emph{Titelbild} als \glossar{Bild-Teil},
\item eine vom \glossar{Urheber} selbst erstellte Beschreibung als
  \glossar{Text-Teil},
\item eine bildnerische Interpretation als weiteren \glossar{Bild-Teil},
\item zugeordnete Merkmale als Menge von URIs aus der
  \glossar{Merkmalsontologie} als Teil der \glossar{Paket-Metainfomation}.
\end{itemize}

\paragraph{Redakteur:} 
Erweiterung der Wordpressrolle „Redakteur“ um die Aufgaben des Freischaltens
von Werken und der Bearbeitung von Meldungen.

\paragraph{Set:} 
Menge von Download-Materialien zu einem \glossar{Projektangebot}.  Einzelne
Materialien können zu mehreren Projektangeboten gehören.  Die Materialien sind
einheitlich im \glossar{Bereitstellungsraum} abgelegt, die Zugehörigkeit zu
einzelnen Projektangeboten wird durch eine Angebotsbeschreibung im RDF-Format
aufgelöst.

Die Beziehung zwischen Projektangeboten und Sets ist vergleichhbar zu der
Beziehung zwischen \glossar{Werken} und \glossar{PVM-Paketen} organisiert.

\paragraph{SKOS:} 
Simple Knowledge Organization System, eine RDF-Ontologie, siehe
\url{http://www.w3.org/2004/02/skos/}.

\paragraph{Text-Teil:}
Spezielle Art von \glossar{Werkteil}, das eine Textdatei als valides
HTML-Fragment enthält.  Um ein einheitliches Erscheinungsbild innerhalb der
\glossar{Plattform} zu garantieren, muss ein Text-Teil spezielle
\emph{Designvorgaben} erfüllen.

\paragraph{Urheber:} 
\glossar{User}, der Zugang zur Werkstatt hat und damit in der Lage ist, eigene
\glossar{Werke} zu erstellen.  Werke können einen Urheber haben, der nicht mehr
als User an der Plattform aktiv ist. Neue Werke können nur von aktiven Users
eingereicht werden. Jedes Werk besitzt einen eindeutigen Urheber.

\paragraph{User:} 
Registrierter und angemeldeter Nutzer der \glossar{Plattform}. Ein User hat die
AGB akzeptiert und hat eine eigene \glossar{Profilseite}. 

Wird der User gelöscht, so wird auch diese Profilseite gelöscht und durch
einen Standardverweis ersetzt. 

Von einem gelöschten User wird der \texttt{foaf:name} und eine Orts-URI als
\texttt{foaf:based\_near} zu Referenzzwecken gespeichert.

\Kommentar{02.03.2016, Gräbe: Eine solche RDF-basierte Menge von Ortsreferenzen
  ist aufzubauen und in die Plattform zu integrieren. }

\paragraph{Werk:} 
Ein \glossar{Paket} speziell unter urheberrechtlichen sowie
plattform-technischen Gesichtspunkten. 

\paragraph{Werkstattprozess:} 
Prozess der Erstellung von Paketinhalten vor deren Veröffentlichung. Dieser
Prozess und die dabei verfügbaren Instrumente und Inhalte sind noch genauer zu
spezifizieren, insbesondere die Frage, in welchem Umfang in diesem
Werkstattprozess auf Teile anderer \glossar{Pakete} zurückgegriffen werden
kann.

\paragraph{Werkteil:}
Bestandteil eines \glossar{Werks}, das eine physische Repräsentation als Datei
hat und so in ein \glossar{Paket} aufgenommen werden kann. Werkteile sind
jenseits des \glossar{Werkstattprozesses} nur über Pakete zugänglich.
\glossar{Pakete} enthalten neben den Werkteilen auch \emph{Beschreibungen}
dieser Werkteile als Metainformationen.

Es gibt \glossar{Bild-Teile} und \glossar{Text-Teile}.

\newpage
\section{Änderungshistorie dieses Dokuments}

\paragraph{29.03.2016, Gräbe:} 
Einarbeitung der Änderungen aus dem Gespräch mit Frau Düwel am 29.03.2016. 

\paragraph{02.03.2016, Gräbe:} 
Einarbeitung der Änderungen wie in den beiden Durchsprachen zum Thema Projekte
am 05.02. und 22.02.2016 mit Frau Düwel abgestimmt.  

\paragraph{15.01.2016, Gräbe:} 
Einarbeitung der Änderungen wie in der Durchsprache am 14.01.2016 mit
Prof. Wendt und Frau Düwel abgestimmt.  Das Glossar wurde weiter gekürzt und
der Abschnitt „FAQ“ einfernt, da er nicht zur Anforderungsaufnahme gehört.
 „FAQ“ einfernt, da er nicht zur Anforderungsaufnahme gehört.

\paragraph{13.01.2016, Gräbe:} 
Es wurde die Strukturierungsform „Formulierungsvorschlag“ einge"|führt und die
Anforderungserhebung noch einmal überarbeitet.

\paragraph{20.12.2015, Gräbe:} 
Eine Reihe von Begriffen wurde weiter konsolidiert und die Requirements
entsprechend dem aktuellen Stand der Anforderungserhebung noch einmal
grundlegend überarbeitet.

\paragraph{17.12.2015, Gräbe:} 
Rollen Redakteur und Moderator wie im Tutorium besprochen im Glossar als
„Redakteur“ zusammengeführt und Beschreibung präzisiert. Beschreibung von
„Profilseite“ und Begriff „Formularseite“ ergänzt. Abschnitt „Konzeptionelle
und Design-Fragen“ ergänzt.

\paragraph{5.12.2015, Gräbe:} 
Der nicht konsolidierte Teil wurde in ein eigenes Dokument „Offene Fragen“
ausgelagert und später ins Wiki übertragen.  Weiter wurden Diskussionen in den
Seminaren am 19. und 26.11. sowie die Ergebnisse der Abstimmung mit Frau Düwel
am 2.12. eingearbeitet.

\paragraph{3.12.2015, Gräbe:} 
Abschnitt „Plattform-Bootstrap“ wurde in das Wiki übertragen und hier entfernt,
da er nichts mit der Anforderungsaufnahme zu tun hat.

\paragraph{12.11.2015, Gräbe:} 
Übernahme der Punkte aus dem Wiki. 

\paragraph{06.11.2015, Gräbe:} 
Diskussionen in den Seminaren am 29.10. und 05.11.  sowie die Absprachen mit
Sebastian Günther am 04.11. eingearbeitet.

\paragraph{17.10.2015, Gräbe:} 
Handreichung Düwel vom 14.10.2015 eingearbeitet und kommentiert.

\paragraph{29.09.2015, Gräbe:} 
Aufzeichnungen zum Gespräch Gräbe -- Düwel eingearbeitet. 

\end{document}
