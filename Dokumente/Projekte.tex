\documentclass[a4paper,11pt]{article}
\usepackage[utf8]{inputenc}
\usepackage{a4wide,url,ngerman}

\newcommand{\Kommentar}[1]{
  \begin{quote}\textbf{Kommentar:} #1 \end{quote}
}

\newcommand{\FV}[1]{
  \begin{quote}\textbf{Formulierungsvorschlag für Einfügung:} #1 \end{quote}
}

\newcommand{\glossar}[1]{{$\to$ \textsc{#1}}}
\parindent0pt
\parskip3pt

\title{Partizipatorisches Virtuelles Museum\\[.6em]
  Kommentierte Version zum Thema „Projekte“}
\date{Version vom 31. Januar 2015}
\author{Hans-Gert Gräbe}

\begin{document}
\maketitle

Dieses Dokument kommentiert die Ausarbeitung \emph{Projekte} von Frau Düwel in
der Version vom 22.01.2016 und setzt die Ausarbeitung in Relation zu weiteren
zum Thema „Projekte“ bereits zusammengetragenen Anforderungen. 

Neue Begriffe, die noch nicht im Glossar erfasst wurden, sind kursiv
ausgezeichnet.

\section{Anmerkung zum Dokument \emph{Projekte} von Frau Düwel} 

\subsection{Was ist ein Projekt?}

\FV{Ein \emph{Ziel} des PVM-Projekts ist es, Lehrenden die Möglichkeit, die
  \glossar{Plattform} im Unterricht zu nutzen.  Neben der Möglichkeit, auf
  Inhalte über unsere allgemeinen Plattformkonzepte zuzugreifen, bieten wir
  Lehrenden die Möglichkeit, eigene \glossar{Projekte} zu verwirklichen oder
  eines unserer \glossar{Projektangebote} mit der eigenen Klasse umzusetzen. }

\subsection{Unterscheidung verschiedener Projekttypen}

In einem \glossar{Projekt} werden Lehrer typischerweise 
\begin{enumerate}
\item \emph{eigene} didaktische Zielstellungen mit den Möglichkeiten unserer
  \glossar{Plattform} umsetzen oder
\item eines unserer \glossar{Projektangebote} für die eigene Zielgruppe nutzen
  und dazu entsprechend modifizieren.
\end{enumerate}

Für die zweite Variante werden unterschiedliche Lehrmaterialien in
verschiedenen Szenarios als \glossar{Projektangebote} im Bereich ME PROJECTS
zur Verfügung gestellt. Ein solches Projektangebot ist eine Sammlung von
ausführlichen Materialien (Informationen zum Thema, zum Inhalt, zu den
Lernzielen, den didaktischen Methoden, zur Zeitplanung, sowie empfohlene
Unterrichtsfächer und Altersklassen, sowie vorgefertigte Arbeitsaufträge mit
Arbeitsblättern) zur Umsetzung eines \glossar{Projekts}.

\Kommentar{In welchem Format liegen diese Materialien vor und wo werden sie
  digital vorgehalten? Brauchen wir einen eigenen Glossareintrag \glossar{Set}
  oder passt das unter \glossar{Projektangebot}? 

  Die Materialien sollten als Open Educational Resources unter einer
  CC-Lizenz zur Verfügung gestellt werden, welche die Nachnutzung \emph{und}
  Modifikation der Materialien erlaubt, um die urheberrechtliche Situation klar
  zu regeln.  Dem müssten dann auch die Autoren der Materialien zustimmen. } 

\Kommentar{Auf welche Weise und in welchem Umfang werden Erfahrungen und
  Materialien von durchgeführten Projekten für die Plattform verfügbar gemacht?
  Wie wird dies insbesondere für Projekte in der ersten Variante realisiert? }

In jeder der beiden Varianten ist vom \glossar{Projektleiter} zunächst ein
\emph{Konzept} für das \glossar{Projekt} zu entwickeln und einzureichen, das
von einem \glossar{Redakteur} daraufhin geprüft wird, ob es mit unserer
\glossar{Plattform} in ihrem aktuellen Ausbauzustand umgesetzt werden kann.

Es gibt zwei Arten von Projekten:
\begin{itemize}
\item [(A)] Projekte, welche nur die auf der Plattform bereits vorhandenen
  Werke nutzen. Mit der Anmeldung und Bestätigung eines solchen Projekts wird
  dem \glossar{Projektleiter} eine \glossar{Projektseite} eingerichtet sowie
  eine \glossar{Ausstellung} mit den Werken im Projektkontext zusammengestellt.
  Die Projektseite kann vom Projektleiter gestaltet werden. Eine Anmeldung der
  Projektteilnehmer an der Plattform ist nicht erforderlich.
\item [(B)] Projekte, in deren Verlauf die Teilnehmer neue Werke erstellen.
  Für derartige Projekte ist eine \glossar{Registrierung} sowohl des
  \glossar{Projektleiters} als auch aller Teilnehmer an der \glossar{Plattform}
  zwingend erforderlich, da nur registrierte \glossar{User} neue
  \glossar{Werke} hochladen können. 

  Es werden nur solche \glossar{Werke} hochgeladen, welche die allgemeinen
  Anforderungen an Werke erfüllen und den standardisierten
  \glossar{Freigabeprozess} der Plattform durchlaufen haben.  Die entstandenen
  \glossar{Werke} werden grundsätzlich dauerhaft in den \glossar{Fundus}
  übernommen.

  Erst nach dieser Freigabe kann der Projektleiter dieses Werk seinem
  \glossar{Ausstellungskontext} zuordnen, dieses also in die projektspezifische
  Ausstellung übernehmen.

  Ein Zurückziehen von Werken aus dem Fundus ist nur nach den allgemeinen
  Richtlinien der PVM-Plattform möglich.

  \Kommentar{Ein solcher Prozess bleibt zu spezifizieren.}

\end{itemize}

Die weiteren Ausführungen konzentrieren sich auf Projekte vom Typ (B), wobei
der \glossar{Projektleiter} in einem realen Lernsetting mit einer Gruppe eine
thematisch gebundene \glossar{Ausstellung} erstellt. Dabei wird jeder
Projektteilnehmer zum \glossar{Urheber} und erstellt und veröffentlicht ein
eigenes \glossar{Werk} zu einen gemeinsamen Thema in der Plattform. Der
\glossar{Projektleiter} fasst die Werke der Gruppe zu einer
\glossar{Ausstellung} zusammen, der gemeinsame Prozess der Werkserstellung wird
auf der Plattform auf einer \glossar{Projektseite} dokumentiert. Jedes Projekt
bekommt dafür eine eigene Seite.

\subsection{Anwenderszenarien}

\subsubsection*{Projekt erstellen und bestätigen}

\begin{itemize}
\item Ein neues \glossar{Projekt} kann nur von einem \glossar{Projektleiter}
  erstellt werden.
\item Der Projektleiter reicht über die \emph{Projektformularseite} einen
  \emph{Antrag} auf ein neues Projekt ein, welcher (wenigstens) Projektkürzel,
  Titel und \emph{Projektkonzept} umfasst. Aus den Metadaten werden
  Projektleiter und Einreichungsdatum ergänzt.

\Kommentar{Die Liste der Angaben in einem Projektantrag ist weiter zu
  konkretisieren. 

  Wie erfolgt der Bezug auf Projektangebote? }

\item Ein solcher \emph{Projektantrag} wird von einem \glossar{Redakteur}
  begutachtet und bestätigt oder verworfen.

\Kommentar{Hier bleibt der Prozess in beiden Ausprägungen weiter zu schärfen. }

\item Für einen bestätigten \emph{Projektantrag} beauftragt der
  \glossar{Redakteur} einen \glossar{Admin}, die Projektseite anzulegen und
  diese in die Menüstrukturen zu übernehmen.

\Kommentar{Die Rechte eines Redakteurs reichen nicht aus, um einen
  Menü"|eintrag anzulegen. Hier ist auch zu bedenken, dass die Angaben aus dem
  Projektantrag in strukturierter Weise in einem \emph{Projektobjekt} auf
  PHP-Ebene erfasst werden. }

\item Nach dem Anlegen dieser Projektinfrastruktur in der Plattform wird der
  Projektleiter informiert und kann sein Projekt im Rahmen der
  \emph{Richtlinien für Projekte} umsetzen. 

\Kommentar{Derartige Richtlinien müssen insbesondere beschreiben, in welcher
  Weise die \glossar{Projektseite} vom \glossar{Projektleiter} zu führen ist. }

\end{itemize}

\subsubsection*{Zuordnen von Personen}

\begin{itemize}
\item Ein \glossar{User} wird \glossar{Projektleiter}, indem er dies bei einem
  \glossar{Admin} beantragt und von diesem der WP-Rolle Projektleiter
  zugeordnet wird.  Ein User kann erst dann einen Projekterstellungsprozess
  starten, wenn er als Projektleiter registriert ist.

  \Kommentar{Dieser Prozess bleibt genauer zu spezifizieren, insbesondere die
    Frage, welche Regelungen Projektleiter einzuhalten haben und wie die
    Verpflichtung auf diese Regeln erfolgt.}

\item \emph{Projektteilnehmer} registrieren sich zunächst als \glossar{User}
  und werden dann über einen ebenfalls zu spezifizierenden Prozess unter
  Vermittlung des \glossar{Projektleiters} einem bereits bestätigten
  \glossar{Projekt} zugeordnet.
\end{itemize}

\subsubsection*{Zuordnen von Werken}

\begin{itemize}
\item Einem \glossar{Projekt} kann nur ein \glossar{Werk} zugeordnet werden,
  das bereits in den \glossar{Fundus} aufgenommen wurde und damit den
  allgemeinen Prozess der \glossar{Freischaltung} von Werken durchlaufen hat.

  Werke von \emph{Projektteilnehmern} müssen diesen Prozess durchlaufen haben,
  bevor sie dem Projekt zugeordnet werden können.

\item Ein dem Projekt zuzuordnendes Werk wird im Zusammenspiel von
  \glossar{Projektleiter} und \emph{Projektteilnehmer} identifiziert und dem
  \glossar{Projekt} und damit der speziellen \glossar{Ausstellung} dieses
  Projektkontexts zugeordnet.

\Kommentar{Auch dieser Prozess bleibt weiter zu spezifizieren.  Insbesondere
  ist zu klären, ob ausschließlich Werke von Projektteilnehmern in den
  Projektkontext übernommen werden können.  Für weitere Werke wäre zu klären,
  welche Rechte der Urheber hat, einer solchen Aufnahme zuzustimmen oder diese
  zu verweigern. Eine klare CC-Lizenz, welche der Allgemeinheit ein einfaches
  Nutzungsrecht einräumt, wäre hier die sauberste Lösung.}

\item Ein Werk kann gleichzeitig in mehreren Projektkontexten referenziert und
  damit in mehreren Ausstellungen gezeigt werden.

\Kommentar{Hierfür ist die Unveränderlichkeit eines \glossar{Werks} im
  \glossar{Fundus} grundlegende Voraussetzung. }
\end{itemize}

\subsubsection*{Löschen von Projekten}

Dieser Punkt ist sehr subtil und in seinen Konsequenzen nicht zu Ende gedacht.
Im Zuge eines \glossar{Projekts} werden neue \glossar{Werke} in den
\glossar{Fundus} aufgenommen sowie neue \glossar{User} angelegt, die dann dem
Projekt als Teilnehmer zugeordnet werden.

Mit dem Löschen des Projekt (und damit auch der \glossar{Projektseite} und des
entsprechenden Menüeintrags) ist zu entscheiden, wie mit diesen
\glossar{Werken} und \glossar{Usern} umzugehen ist.

Für jedes einzelne \glossar{Werk} ist vor dem Löschen zu prüfen, ob es in
anderen Projektkontexten, \glossar{Projektangeboten} oder
\glossar{Ausstellungen} referenziert wird, und diese Referenzen sind vorab zu
löschen.  Dabei ist zu prüfen, ob diese Löschung den jeweiligen Kontext
entwertet und vorab das entsprechende \glossar{Projektangebot} aus der
\glossar{Plattform} zu entfernen.

Da jeder Urheber User sein muss, können nur solche \glossar{User} gelöscht
werden, die keine \glossar{Urheber} sind. Auch dies ist vor dem Löschen eines
Users zu prüfen und ggf. weitere Werke aus der Plattform zu entfernen.

Derartige Prozesse sind im Plattformdesign bisher grundsätzlich nicht
berücksichtigt.

\Kommentar{In den bisherigen Requiremenents war dieser Punkt wie folgt
  formuliert: Projektleiter erstellen und verwalten Projekte im Rahmen ihrer
  Möglichkeiten und Intentionen. Wenn der Projektleiter sein Projekt löscht, so
  wird bei allen Werken, die bisher diesem Projekt zugeordnet waren, lediglich
  die Projektzugehörigkeit gelöscht, nicht jedoch die Werke selbst.}

\subsection{Struktur und Seitenaufbau}

Ist „Alle Projekte“ eine Übersichtsseite wie „Projektangebote“?  Die Einordnung
der Projekt-Einzelseiten und der Projektangebots-Einzelseiten in die
Menüführung ist nicht dargestellt.

Die Menüstruktur wird sehr unübersichtlich, wenn dies bis in die dritte
Menü-Ebene (Einzelseiten Projekt und Projektangebot) propagiert wird und dort
mit Blick auf die darzustellende Vielzahl potenziell umfangreiche
Menüstrukturen eingebunden sind. Es ist zu prüfen, ob ein solches Design
wirklich sinnvoll ist und welche alternativen Möglichkeiten der Nutzerführung
es gibt. 

Die Seite ME PROJECTS enthält sehr viel Informationen, „Projekt erstellen“ wird
auch noch einmal als Menüpunkt angeboten. Unter „Projekt erstellen“ könnte
durch Abfrage der Rolle des Users abgeprüft werden, ob dieser bereits
Projektleiter ist und ggf. dort zunächst das Formular „Projektleiter werden“
angezeigt werden.

Seite „Alle Projekte“: Hier werden weitere Informationen zu einem Projekt
vorausgesetzt (Institution, Profilbild [von wem?]). Diese sind beim
Projektantrag mit abzufragen.

Die Struktur eines Projektangebots (Set) ist noch genauer zu klären
einschließlich der Frage, wie die verschiedenen Medien (Arbeitsblätter,
Handreichungen usw.)  vorgehalten werden und ob auch die detaillierte
Beschreibung des Projektangebots etwa im pdf-Format vorgehalten werden sollte
und nur eine strukturierte Metainformation zum Projektangebot in die Plattform
eingearbeitet wird.  Dies würde die weitere Pflege der Plattform und die
Aktualisierung der Unterlagen von Projektangeboten entkoppeln und damit die
Prozesse um die Plattform herum vereinfachen.  Dies entspricht der im Bereich
semantischer Technologien üblichen Unterscheidung zwischen Ressourcen und
Ressourcenbeschreibungen. Dazu wäre aber zu klären, welche Unterlagen zu einem
Projektangebot gehören und wie diese technisch zu einem „Bündel“ geschnürt
werden können.

\subsection{Bisherige Glossar-Einträge} 

\paragraph{Projekt:}
Konkrete Umsetzung eines \glossar{Projektangebots}.

\paragraph{Projektangebot:}
Spezifischer didaktischer Rahmen zur Nutzung der Plattform. Im Rahmen des PVM
werden Projektangebote systematisch entwickelt.

\paragraph{Projektleiter:}
Eine Erweiterung der WP-Rolle „Mitarbeiter“ um das Recht, Projekte anzulegen
und zu verwalten.  Jedes \glossar{Projekt} verfügt über einen Projektleiter.

\subsection{Weitere Fragen aus den nicht konsolidierten Anforderungen} 

Einbindung in die Menüstruktur der Plattform:
\begin{itemize}
\item Untermenü mit Einträgen Tutorial, Handreichungen (darin verschiedene
  \glossar{Projektangebote}, z.B. ME TALES, gut als Icons wie Logo
  visualisierbar) 

  \Kommentar{Jeder dieser Unterpunkte muss mit einem eigenen Seitenkonzept
    untersetzt sein.}
\item Welche inhaltlichen und strukturellen Vorgaben für
  \glossar{Projektseiten} als Übersichtsseiten zum Projekt gibt es? Wie ist
  dort das (eingereichte und abgestimmte) \emph{Projektkonzept} darzustellen? 
\item Welche weiteren Materialien in welchen Formaten können einem
  \glossar{Projekt} zugeordnet werden? Wie werden diese Materialien evaluiert
  und wie wird gesichert, dass diese Materialien in den allgemeinen PVM-Kontext
  passen? 
\item Wie wird zwischen Projekten in Realisierung und bereits realisierten
  Projekten unterschieden?
\item Welche Informationen sind bei bereits realisierten Projekten
  darzustellen? 
\item Kann es zu einem \glossar{Projektkonzept} mehrere realisierte
  \glossar{Projekte} geben? Wie soll dieser Zusammenhang dargestellt werden? 
\end{itemize}

Erste Vorstellungen für einen Workflow zum Einreichen und Begutachten von
\glossar{Projekten}:
\begin{itemize}
\item \glossar{User} beantragt Rolle \glossar{Projektleiter} und wird dazu von
  einem \glossar{Admin} freigeschaltet.
\item \glossar{Projekt} wird mit einem Formular beantragt. Das
  \glossar{Projekt} wird nach redaktioneller Begutachtung einer kleinen
  Projektbeschreibung (Bewerbung) von einem \glossar{Redakteur} freigeschaltet.

  \Kommentar{ Dazu ist der Freigabeprozess genauer zu spezifizieren.}

\item Der Projektleiter kann nun seinem \glossar{Projekt} mehrere
  \glossar{Werke} zuordnen. 

\Kommentar{Das bezieht sich auf Projekte vom Typ~(A)? }

\item Bestätigte \glossar{Projekte} bekommen eine eigene \glossar{Projektseite}
  auf der Plattform, die vom \glossar{Projektleiter} nach noch festzulegenden
  Vorgaben gestaltet werden kann und im Untermenü Projekte erscheint,
  z.B. „Projekt Bleistifte“.

\Kommentar{Dann muss das Untermenü anders gestaltet sein als dies oben
  vorgeschlagen wird.}
\end{itemize}


\end{document}
