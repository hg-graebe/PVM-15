\documentclass[a4paper,11pt]{article}
\usepackage[utf8]{inputenc} 
\usepackage{a4wide,url,ngerman}

\newcommand{\glossar}[1]{{$\to$ \textsc{#1}}}
\parindent0pt
\parskip3pt
\setcounter{tocdepth}{2}

\title{Partizipatorisches Virtuelles Museum\\[.6em] 
  Beschreibung der Umsetzung}
\date{Version vom 28. Juli 2016}
\author{Sebastian Günther, Hans-Gert Gräbe}

\begin{document}
\maketitle

\tableofcontents

\newpage  

Kern des \emph{Partizipatorischen Virtuellen Museums} (PVM) ist eine digitale
\glossar{Plattform}, die auf Beteiligung von Internetusern ausgerichtet ist.
Das PVM ist eine virtuelle Ausstellungsform mit festgelegter Thematik und
Präsentationsart, die konkrete inhaltliche Ausprägung entsteht jedoch
ausschließlich durch die \glossar{Autoren} und deren \glossar{Werke}.

Ein Ziel des PVM-Projekts ist es, Lehrenden die Möglichkeit zu geben, die
\glossar{Plattform} für den Unterricht zu nutzen.  Neben der Möglichkeit, auf
Inhalte entsprechend des allgemeinen Plattformkonzepts zuzugreifen, wird
Lehrenden die Möglichkeit geboten, eigene \glossar{Projekte} zu verwirklichen
oder eines der \glossar{Projektangebote} mit der eigenen Klasse umzusetzen.

\emph{Ziel} dieses Dokuments ist es, die Umsetzung der zur Erstellung und
Weiterentwicklung der \glossar{Plattform} spezifizierten Anforderungen zu
beschreiben.

Mit \glossar{} versehene Worte verweisen auf Einträge im Glossar. 

\section{Konzeptionelle Grundlagen}

\subsection{Allgemeines}\label{grundlagen.allgemeines}

Das PVM-Konzept geht von der Grundvorstellung (Metapher) eines
Museums\footnote{Der Museumsbegriff wird im Kontext des PVM lediglich als
  \emph{Analogie} bzw.\ Vergleich verwendet, um zu erklären oder Vorgänge zu
  beschreiben, die einem Museum ähnlich sind, z.\,B.\ das Präsentieren von
  Objekten (etwa im About-Text oder im Tutorial).  In der Außendarstellung und
  als Bezeichnung für die Site soll der Museumsbegriff nicht verwendet werden,
  da Jugendliche und junge Erwachsene dem \emph{traditionellen} Museum eher
  ablehnend gegenüberstehen (vgl. Bocatius 2014).  Mit der Formulierung „Kern
  des \emph{Partizipatorischen Virtuellen Museums} (PVM) ist eine digitale
  {Plattform}“ wird dem Umstand Rechnung getragen, dass das PVM-Konzept über
  die Verfügbarkeit einer digitalen Plattform als virtuelle Ausstellungsform
  hinausreicht und als „Museum“ weitergehende didaktische und konzeptionelle
  Ziele verfolgt, insbesondere mit der Vorgabe von Design-Richtlinien sowie
  der Möglichkeit zu \glossar{Projekten}.} aus, in dem öffentlich zugängliche
\glossar{Werke} als fertige, nicht weiter modifizierbare Einheiten in einer
„Sammlung“ zusammengefasst sind.  Auf der \glossar{Plattform} werden derartige
von \glossar{Autoren} nach plattformweit gültigen Regeln bereitgestellte
digitale Darstellungen von Gegenständen gesammelt. Die Gegenstände werden
dabei repräsentativ als Darstellungsmittel der eigenen Person betrachtet.
Voraussetzung ist, dass der Gegenstand etwas mit der eigenen Person zu tun
hat.  Dann kann er diese repräsentieren.

\subsubsection{Besucher und User}

Der Schwerpunkt liegt auf der Sammlung und Reihung einer größeren Anzahl
derartiger \glossar{Werke}, die zwar nach einheitlichen Regeln erstellt
wurden, um ein plattformweites einheitliches Design zu gewährleisten, die aber
unterschiedlich zusammengestellt und gereiht werden können und so verschiedene
Sichten auf den gesamten Korpus erlauben.  Dazu können die Werke auf der
\glossar{Plattform} von (nicht angemeldeten) \glossar{Besuchern} über Such-
und Filterfunktionen auf der Basis zugeordneter \glossar{Merkmale} verschieden
strukturiert werden.

Die Nutzung weitergehender partizipativer Möglichkeiten setzt eine
Registrierung an der Plattform als \glossar{User} voraus.  User sind zur
interaktiven Teilnahme aufgefordert, können eigene Werke einreichen, Werke
bewerten und persönliche Präferenzen deutlich machen.

\subsubsection{Werke}

\glossar{Werke} werden von einzelnen Personen, den \glossar{Autoren}, erstellt,
nach der Erstellung einem Begutachtungsprozess durch \glossar{Redakteure}
unterzogen und nach einem positiven Votum in den \glossar{Fundus} übernommen.

Als dingliche Einheit liegen \glossar{Werke} in der Form von \glossar{Paketen}
zusammengehörender \glossar{Werkteile} vor, die einer vorgegebenen
einheitlichen Strukturierung folgen und zugleich
\begin{itemize}\itemsep0pt
\item Ergebnis des kreativen Schaffens des \glossar{Autors} im Sinne der
  PVM-Konzeption eines reflexiven Zugangs zum Selbst über dingliche
  Darstellungsmittel sowie
\item Ausgangspunkt rezeptiven Zugangs (anonymer) \glossar{Besucher} zu einer
  vernetzten Vielfalt solcher Artefakte sind.
\end{itemize}

\glossar{Werke} bestehen aus
\begin{itemize}\itemsep0pt
\item Titel, 
\item Titelbild als \glossar{Bild-Teil},  
\item einer Assoziation dazu als \glossar{Text-Teil}, 
\item einer bildnerischen Darstellung als weiterem \glossar{Bild-Teil} sowie
\item Merkmalszuweisungen aus einer vorgegebenen Menge von
  \glossar{Merkmalen}. 
\end{itemize}
Die \glossar{Merkmale} sind als morphologische Tabelle plattformweit
vorgegeben.  Zu jedem Werk ist weiter dessen \glossar{Autor} sowie das
Einreichungsdatum hinterlegt.  Über den \glossar{Autor} ist eine
Orts-Information verfügbar.

Neue Werke können grundsätzlich nur von registrierten \glossar{Usern}
hochgeladen werden.  Ein \glossar{User} kann dabei als Stellvertreter für
mehrere \glossar{Autoren} agieren, etwa ein Lehrer als Stellvertreter für
seine minderjährigen Schüler.  Ein \glossar{User} muss zwingend an der
Plattform registriert sein, ein \glossar{Autor} dagegen nicht.

Vor dem Hochladen eines \glossar{Werks} müssen dessen Einzelteile am eigenen
Computer erstellt bzw. zusammengestellt werden.  Das Hochladen fertiger
\glossar{Werke} folgt dem allgemeinen Einreicheprozess der Plattform und wird
durch Upload-Formulare unterstützt.  Es können nur solche Werke hochgeladen
werden, welche die allgemeinen Anforderungen an Werke erfüllen. Werke werden
nach Durchlaufen des standardisierten \glossar{Freigabeprozesses} der Plattform
grundsätzlich dauerhaft in den \glossar{Fundus} übernommen.  

Veröffentlichte Werke können nicht weiter verändert werden.  Ein Löschen von
veröffentlichten Werken aus dem Fundus ist nur in Ausnahmefällen und nur nach
den allgemeinen Richtlinien der PVM-Plattform möglich.

\subsection{Projekte}\label{grundlagen.projekte}

\subsubsection{Projekte und Projektangebote}

In einem \glossar{Projekt} werden Lehrer typischerweise 
\begin{enumerate}\itemsep0pt
\item \emph{eigene} didaktische Zielstellungen mit den Möglichkeiten der
  \glossar{Plattform} umsetzen oder
\item eines der \glossar{Projektangebote} für die eigene Zielgruppe nutzen und
  dazu entsprechend modifizieren.
\end{enumerate}
Es ist prinzipiell möglich, \emph{freie Projekte} auf der Basis der
\glossar{Plattform} umzusetzen.  Details hierzu sind mit den
Plattformbetreibern in jedem Fall projektbezogen abzustimmen.

Als \emph{Hauptform} wird jedoch darauf orientiert, ein Projekt auf der
Grundlage eines der \glossar{Projektangebote} umzusetzen. Dafür werden
unterschiedliche Lehrmaterialien in verschiedenen Szenarien als \glossar{Sets}
im Bereich ME PROJECTS zur Verfügung gestellt. 

Ein solches Set ist eine Sammlung ausführlicher Materialien (Informationen zum
Thema, zum Inhalt, zu den Lernzielen, den didaktischen Methoden, zu
Zeitplanung, empfohlenen Unterrichtsfächern und Altersklassen sowie
vorgefertigte Arbeitsaufträge mit Arbeitsblättern) zur Umsetzung eines
Projekts und enthält in der Regel
\begin{itemize}\itemsep0pt
\item einen Erklärtext mit Akzentuierung, Altersgruppe, übergreifendes Fach,
\item einen Text zur Beschreibung,
\item eine Grob-Übersicht (Zim-Plan) mit Ziel, Inhalt, Methode, Zeit (Tabelle,
  vier Einheiten für die Doppelstunde),
\item eine Stundenplanung (Tabelle) mit Ziel, Inhalt, Methode, Zeit (Tabelle
  mit Literaturhinweisen),
\item Methoden mit Glossar sowie 
\item Arbeitsblätter und Materialien (Medien).
\end{itemize}
Diese Materialien werden als komplettes Set von Materialien zum Download
vorgehalten.  Einzelne Materialien können zu mehreren \glossar{Sets} gehören,
die Zugehörigkeit zu einzelnen Projektangeboten wird durch eine
\glossar{Angebotsbeschreibung} aufgelöst.  Das Kernteam des PVM-Projekts
kümmert sich um die Weiterentwicklung dieser Materialien, wozu insbesondere
das Feedback der Projekte ausgewertet wird.  Die Erhebung dieses Feedbacks
erfolgt in direktem Kontakt mit den \glossar{Projektleitern}.

\subsubsection{Projekte umsetzen}

Zunächst ist vom \glossar{Projektleiter} ein \emph{Konzept} für das Projekt zu
entwickeln und einzureichen, das von einem \glossar{Redakteur} daraufhin
geprüft wird, ob es mit der \glossar{Plattform} in ihrem aktuellen
Ausbauzustand umgesetzt werden kann. Nach der Freigabe kann das Projekt
beginnen. 

Ein Projekt durchläuft die Phasen \emph{Beantragung}, \emph{Durchführung},
\emph{Archiv}.  Nach Beantragung und Freigabe wird zur Durchführung des
Projekts auf der Plattform eine \glossar{Projektseite} eingerichtet, die vom
Projektleiter entsprechend der \emph{Richtlinien für Projekte} geführt und
gestaltet werden kann.

Diese \glossar{Projektseite} ist während der Durchführungsphase als Direktlink
(öffentlich) erreichbar, aber nicht in die Menüstrukturen des PVM eingebunden.
Dies erfolgt erst nach Abschluss des Projekts und Übergabe der Seite durch den
\glossar{Projektleiter} an einen \glossar{Redakteur}, der die Einbindung einer
Projektkachel in die Seite „Alle Projekte“ veranlasst.  Damit soll
gewährleistet werden, dass die Projektseite sowohl als Arbeitsseite während der
Projektdurchführung als auch als Ergebnisseite nach Projektende verwendet
werden kann und zugleich die Einheitlichkeit des Plattformdesigns gewährleistet
ist.

\subsubsection{Projekte und Werke}

Im Projektverlauf erstellen die Teilnehmer in dem Regelfall neue Werke im
Kontext eines thematisch gebundenen realen Lernsettings. Dabei wird jeder
Projektteilnehmer zum \glossar{Autor}.

Das Einstellen der so erstellten Werke in die \glossar{Plattform} kann entweder
direkt von den \glossar{Autoren} vorgenommen werden, die dazu als
\glossar{User} an der Plattform registriert sein müssen, oder durch andere
qualifizierte \glossar{User} (etwa den Projektleiter oder weitere Tutoren), die
im Namen der Autoren agieren, entsprechende Autorenprofile einrichten sowie die
Werke hochladen und den entsprechenden Autoren zuordnen.  Das Hochladen von
Werken folgt dabei dem allgemeinen \glossar{Einreichungs}- und
\glossar{Freigabe}prozess.  Erst nach dieser Freigabe kann ein Werk einem
Projektkontext zugeordnet und damit in die projektspezifische Ausstellung
übernommen werden.

Die Zuordnung von \glossar{Werken} zu einem \glossar{Projekt} erfolgt in einem
zweistufigen Prozess, in dem die jeweils Verantwortlichen (Autoren oder User)
das Werk zunächst für das Projekt vorschlagen und danach der
\glossar{Projektleiter} entscheidet, ob die vorgeschlagenen Werke seinem
Projekt zugeordnet werden. Ein Werk kann mehreren Projekten zugeordnet sein. 

Der \glossar{Projektleiter} dokumentiert den gemeinsamen Prozess der
Werkserstellung auf einer \glossar{Projektseite}.

\subsection{Das PVM-Rollenkonzept}\label{grundlagen.rollenkonzept}

Das PVM-Rollenkonzept nutzt die WP-Standardrollen \emph{Betreuer},
\emph{Redakteur} und \emph{Administrator}, denen die folgenden grundlegenden
Aufgabenbereiche zugeordnet sind: 
\begin{quote}
  \begin{tabular}{l|p{.65\textwidth}}
    \textbf{Rolle} & \textbf{Aufgaben}\\\hline
    Administrator & Konfiguration und Anpassung der \glossar{Plattform} sowie
    Verwaltung von Accounts und deren Rollenzuordnung.\\ 
    Redakteur & Freigabe von \glossar{Projekten}, Freischalten von
    \glossar{Werken}, Bearbeiten von Missbrauchsmeldungen.\\ 
    Benutzer & Registrierter \glossar{User}, der eigene \glossar{Werke}
    einreichen kann.\\  
    Besucher & Anonymer Nutzer des öffentlichen Bereichs der
    \glossar{Plattform}.
  \end{tabular}
\end{quote}
Weitere Rollen wie \glossar{Projektleiter} sind als Bündel zusätzlicher Rechte
gestaltet, die mit jeder WP-Rolle kombiniert und damit jedem \glossar{User}
von einem \glossar{Admin} über eine \glossar{Profilseite} des
Users\footnote{Dabei ist zu beachten, dass einem User mehrere Autoren und
  damit auch mehrere Profilseiten zugeordnet sein können. Die Rechte des Users
  können über \emph{jede} dieser Profilseiten geändert werden.} zugewiesen
werden können. Aktuell könne die folgenden Rechte vergeben werden:
\begin{itemize}\itemsep0pt
\item fremde Werke bearbeiten,
\item fremde Werke verwalten,
\item Werke veröffentlichen,
\item Projekte erstellen,
\item Projekte freischalten,
\item andere Benutzer zu Projekt einladen,
\item unter anderen Autorenprofilen hochladen.
\end{itemize}

Der öffentliche Bereich des Museums kann grundsätzlich von jedem
\glossar{Besucher} genutzt werden. Ein Besucher kann sich im Museum umschauen
und dabei den verschiedenen Filter- und Navigationspfaden folgen, aber selbst
keine Änderungen am Zustand des Museums vornehmen.

Jede weitergehende Nutzung der \glossar{Plattform} setzt die Registrierung als
\glossar{User} voraus und ist nur im angemeldeten Zustand möglich.

\subsection{User und Autoren}

Die Rolle \glossar{User} ist der WP-Rolle „Benutzer“ zugeordnet. Nur ein
\glossar{User} kann ein neues \glossar{Werk} als \glossar{Autor} einreichen,
allerdings kann es Autoren geben, die nicht oder nicht mehr als User an der
Plattform registriert sind.

Jeder Autor hat eine eigene \glossar{Profilseite}. Ist der Autor zugleich User,
so wird diese Profilseite um weitere user-spezifische Informationen (etwa seine
Favoriten unter den Werken) ergänzt.  Zu einem nicht mehr an der Plattform
aktiven Autor wird nur ein standardisiertes Autorenprofil angezeigt.

Autorenprofile können auf Antrag an einen \glossar{Admin} und nur zusammen mit
den \glossar{Werken} dieses Autors von der Plattform gelöscht werden.

\subsection{Werke}

Das \glossar{Einreichen} eines Werks wird durch eine \glossar{Formularseite}
unterstützt, über welche die Einzelteile zusammengetragen und als
\glossar{Werk} gespeichert werden können.

Das Werk erscheint nicht direkt im Frontend der \glossar{Plattform}, sondern
muss dafür durch einen \glossar{Redakteur} \glossar{freigegeben} werden.

Ein freigegebenes Werk kann danach einem oder mehreren \glossar{Projekten}
zugeordnet werden.

\subsection{Entwicklungs- und Betreiberkonzept}
\label{grundlagen.betreiberkonzept}

Die \glossar{Plattform} basiert auf einer Standard-Wordpress-Installation mit
einem eigenen \textbf{PVM-Plugin}, in dem plattformspezifische Funktionalitäten
gebündelt werden.

Das Design der Plattform wird über ein eigenes \textbf{PVM-Theme} realisiert,
das als Child Theme des Basisthemes
\emph{Esteeme}\footnote{\url{https://wordpress.org/themes/esteem/}} realisiert
ist.  Bei der Auswahl des Basisthemes wurde darauf geachtet, dass es
\emph{responsive Design} unterstützt.

Die Plattform kann damit einfach auf einem anderen Server entsprechend dem
vorliegenden \textbf{Relokationskonzept} neu ausgerollt werden.

\subsubsection{Rollen im Betriebskonzept} 

Zum Betrieb der Wordpress-Produktivinstanz der PVM-Plattform sind folgende
Rollen zu besetzen:
\begin{itemize}
\item \emph{Site-Admin} --- zuständig für die Installation und Aktualisierung
  der grundlegenden Software auf dem Rechner, das Basis-Backup sowie Kontakt
  bei Rechnerausfällen und Neustart.
\item \emph{Portal-Admin} --- Aktualisierung der Wordpress-Instanz auf neue
  Versionen, Installation und Aktualisierung von Plugins und des PVM-eigenen
  Themas, Upload-Verwaltung.  Ein Portal-Admin benötigt Zugriff auf das
  Installationsverzeichnis auf dem Server.
\item \emph{WP-Admin} --- zuständig für konzeptionelle Fragen sowie die
  Konfiguration der verschiedenen Anwendungen, Plugins und Funktionen über die
  Admin-Konfigurations-Oberfläche von Wordpress. Ein WP-Admin benötigt keinen
  Zugriff auf das Installationsverzeichnis auf dem Server.
\item \emph{WP-Redakteure} und \emph{WP-Autoren} --- zuständig für die Inhalte
  der Plattform.
\end{itemize}

Über Ressourcen für die Weiterentwicklung der Plattform sowie ein dauerhaftes
Betriebskonzept ist noch nicht entschieden. 

  

\section{Anwenderszenarien}

\subsection{Möglichkeiten für nicht angemeldete Besucher} 

\subsubsection*{Werke suchen und zusammenstellen}

Die Such- und Filterfunktion verfügt über zwei grundsätzliche Funktionen. Dies
ist zum einem die \textbf{Volltextsuche}, die bei allen \glossar{Werken} im
Titel und im dazugehörigen Text nach Übereinstimmungen zur Suchphrase
sucht\footnote{Das ist aktuell implementiert, aber noch nicht im Frontend
  eingebunden -- die aktuelle Suche erfolgt nur über die WP-eigenen Seiten und
  Beiträge, die Werktexte werden \emph{nicht} erfasst.}.

Zum anderen gibt es eine \textbf{Filtermöglichkeit nach \glossar{Merkmalen}},
welche nur diejenigen \glossar{Werke} zurückgibt, welche mit der ausgewählten
Merkmalskombination markiert sind.

\subsection{Möglichkeiten für angemeldete Besucher (\glossar{User})}

\subsubsection{Registrierung und Authentifizierung}

Für eine Registrierung sind die folgenden Angaben obligatorisch\footnote{Das
  muss bei der Ablösung von \texttt{wp-members} noch geklärt werden.
  Informationen, die auf der \glossar{Profilseite} erscheinen sollen, sind
  nicht nur von \glossar{Usern}, sondern von allen \glossar{Autoren} zu
  erheben. }:
\begin{itemize}
\item obligatorisch: Name, Benutzername (nick name), Email, Standort 
\item optional: eigenes Foto, Verweis auf eigene Webpräsenz 
\end{itemize}
Der \textbf{Standort} kann aus einer plattformweit vorgegebenen Liste von
Standorte ausgewählt werden, die durch einen Portal-Admin erweitert werden
kann.

Im Zuge der Registrierung als \glossar{User} sind die AGB zu bestätigen.  Das
Datum der Registrierung wird (nach WP-Standard) im Nutzerprofil gespeichert. 

\subsubsection{Benutzerprofil und \glossar{Profilseite}}

Das \emph{WP-Nutzerprofil} wird vom WP-Plugin \texttt{wp-members}\footnote{Es
  ist geplant, dieses Plugin abzulösen.} verwaltet und kann nach Anmeldung im
WP-Backend bearbeitet werden. Dieses Nutzerprofil ist nicht öffentlich. 

Jeder \glossar{User} ist zugleich potenzieller \glossar{Autor} und hat damit
eine \glossar{Profilseite} als Autor, auf der neben den Angaben zum
\glossar{Autor} (Name, Ortsangabe, perspektivisch ein Profilbild) dessen
\glossar{Werke} sowie dessen Favoriten\footnote{Favoriten und Bewertungen
  haben nur \glossar{User}, keine \glossar{Autoren}.} gelistet sind.

Die Angaben im Autorenprofil sowie die Werke des Autors bleiben auch nach dem
Löschen des Accounts in der Plattform erhalten, um Autorenschaften
referenzieren zu können.  Der \glossar{User} kann in Ausnahmefällen zusätzlich
beim \glossar{Admin} beantragen, dass mit dem Löschen des Accounts auch
einzelne oder alle zugeordneten Autorenprofile gelöscht werden.  Dies ist nur
zusammen mit den zugehörigen \glossar{Werken} möglich. In diesem Fall hat der
Admin die zusätzlichen Aktivitäten nach Abschnitt \ref{werk.loeschen} in die
Wege zu leiten.

\subsubsection{Werke bewerten}

Ein \glossar{User} findet auf jeder Einzelseite eines Werks die Möglichkeit,
für dieses Werk Bewertungen aus einer plattformweit vorgegebenen Liste von
Bewertungsmöglichkeiten abzugeben oder diese zu modifizieren. 

\subsubsection{Mitarbeit in einem Projekt}

\glossar{User} können in einem \glossar{Projekt} mitarbeiten.  Eine spezifische
Einladung ist hierfür nicht erforderlich. Die Mitarbeit im Projekt orientiert
sich an den \emph{projektspezifischen Zielen} und erfolgt unter Anleitung und
Koordinierung des \glossar{Projektleiters}.

Im Rahmen einer solchen Mitarbeit können eigene bereits freigegebene
\glossar{Werke} dem Projekt zugeordnet werden. Dies ist durch den
\glossar{Autor} des Werks zu beantragen und durch den \glossar{Projektleiter}
zu bestätigen.

Werke können mehreren \glossar{Projekten} zugeordnet sein. 

\subsubsection{Löschen eines Users}\label{nutzer.loeschen}

Ein \glossar{User} kann bei einem \glossar{Admin} beantragen, dass sein Account
gelöscht wird.  

Das Löschen eines Useraccounts tangiert die zugeordneten
\glossar{Profilseiten} von Autoren und die zugehörigen \glossar{Werke} nicht.
Diese verbleiben standardmäßig weiter in der Plattform, im Autorenprofil
werden aber user-spezifische Informationen (wie etwa Bewertungen von Werken)
nicht mehr angezeigt, da diese zusammen mit dem Useraccount gelöscht werden.

Der User kann in Ausnahmefällen zusätzlich beim \glossar{Admin} das Löschen
einzelner oder aller ihm zugeordneter Autorenprofile beantragen. In diesem
Fall hat der Admin zusätzliche Aktivitäten nach Abschnitt \ref{werk.loeschen}
in die Wege zu leiten.

\subsection{Workflow für Werke}

Ein Werk befindet sich stets in einem der folgenden Zustände:
\begin{itemize}\itemsep0pt
\item „in Bearbeitung“ -- Werk wird vom Autor bearbeitet,
\item „in Begutachtung“ -- Werk ist zur Freigabe angemeldet,
\item „freigegeben“ -- Werk ist freigegeben,
\item „gelöscht“ -- Werk ist gelöscht.
\end{itemize}
In jedem der Zustände ist das Werk in der Plattform verfügbar, der Zustand
steuert allein die Sichtbarkeit und Bearbeitbarkeit des Werks durch Personen
in verschiedenen Rollen. Nur ein gelöschtes Werk kann komplett aus der
Plattform entfernt werden.

Ein Werk wird grundsätzlich über den ins Frontend integrierten Editor
(Edit-Template) erstellt. Das kann entweder als „Neues Werk erstellen“ (leeres
Formular) geschehen oder per Hochladen eines PVM-Pakets im zip-Format aus
externer Quelle. Im zweiten Fall wird das Formular entsprechend
vorinitialisiert. 

Im Zuge des Editierprozesses werden
\begin{itemize}\itemsep0pt
\item eine Werk-Id vergeben und das Zustandsflag gesetzt,
\item Werkdaten in die Datenbank eingetragen,
\item Bilder entsprechend zugeschnitten und abgespeichert,
\item der Text in die Datenbank übernommen,
\item Merkmale entsprechend der hinterlegten morphologischen Tabelle
  zugeordnet und
\item das Autorenprofil zugeordnet.
\end{itemize}
Über die Autoren-Id ist eine User-Id zu erreichen, wenn der Autor als User an
der Plattform existiert. Über diese User-Id werden weitere im Profil
hinterlegte Informationen ausgelesen, die dann -- neben einer Liste der eigenen
Werke sowie der eigenen Bewertungen anderer Werke -- auf der Profilseite des
Autors angezeigt werden.

Werke sind in dieser Phase im Zustand „in Bearbeitung“. Alle Werke eines Autors
in diesem Zustand werden im Backend unter „Meine Werke“ gelistet,
\begin{itemize}
\item mit einem Link „Editieren“, über den das Edit-Formular gestartet werden
  kann,
\item mit einem Link „Veröffentlichen“, womit das Werk zur Veröffentlichung
  eingereicht wird und damit in den Zustand „in Begutachtung“ übergeht,
\item mit einem Link „Löschen“, mit dem das Werk in den Zustand „gelöscht“
  versetzt werden kann (das Werk ist noch da, aber als gelöscht markiert und
  für den Autor nicht mehr zugänglich).
\end{itemize}
Im Zustand „in Begutachtung“ kann das Werk nicht mehr vom Autor editiert
werden.  Das Werk erscheint in der Redaktionsliste „Werke zur Freigabe“ und
kann von einem Redakteur
\begin{itemize}
\item über den Link „Vorschau“ in einer Vorschau betrachtet werden (dazu wird
  das Seitentemplate „Einzelnes Werk“ mit der jeweiligen Werk-Id aufgerufen),
\item über den Link „Werk editieren“ redaktionell überarbeitet werden (dazu
  wird das Edit-Template für Werke gestartet),
\item über den Link „Freischalten“ in den Fundus übergeben werden, womit das
  Werk in der Zustand „veröffentlicht“ übergeht und in diesem Zustand weder
  vom Autor noch von der Redaktion bearbeitet werden kann,
\item über den Link „Löschen“ in den Zustand „gelöscht“ überführt werden.
\end{itemize}
Beim Übergang in den Zustand „veröffentlicht“ wird zusätzlich eine zip-Datei
im PVM-Format für den Export des Werks angelegt.

Ein Werk im Zustand „gelöscht“ kann komplett mit allen zugehörigen Dateien aus
der Plattform entfernt werden. Bis dahin ist das Werk noch über seine Werk-Id
zu erreichen.

\subsubsection{Ein neues Werk zusammenstellen und einreichen}

Ein \glossar{User} kann ein neues \glossar{Werk} formularbasiert
zusammenstellen und zur \glossar{Freigabe} anmelden.  

Einem User kann ein Autorenprofil oder -- bei entsprechenden Rechten -- auch
mehrere Autorenprofile zugeordnet werden.  Mit jedem Autorenprofil ist eine
\glossar{Profilseite} verbunden.  Ein \glossar{Werk} ist stets mit genau einem
Autorenprofil fest verknüpft. Die Zuordnung kann nicht verändert werden.

Über die Formularseite ist es möglich, ein \glossar{Werk} in Etappen zu
erstellen. Der jeweilige Arbeitsstand wird dabei gespeichert. Eine Übersicht
zeigt an, welche \glossar{Werkteile} bereits hochgeladen wurden und welche
Teile noch fehlen.

\subsubsection{Freigabe eines Werks}

Nur vollständige Werke können zur \glossar{Freigabe} angemeldet werden und
gehen dabei in den Zustand „in Begutachtung“ über.  

In diesem Zustand „in Begutachtung“ kann das Werk vom Autor nicht mehr
verändert werden, ist aber auch noch nicht in den \glossar{Fundus} übernommen
und damit noch nicht in der Übersicht „Alle Werke“ zu sehen.

Die \glossar{Freigabe} erfolgt durch einen \glossar{Redakteur}, der das Werk
nach entsprechender Prüfung für den \glossar{Fundus} freigibt (Zustand
„freigegeben“), das Werk an den Autor zurückgibt (Zustand „in Bearbeitung“)
oder löscht (Zustand „gelöscht“) . Die Freigabe durch einen Redakteur soll in
erster Linie das Einhalten der AGB's (die jeder \glossar{User} beim Erstellen
seines Accounts bestätigt) sowie eine saubere Zuordnung gewährleisten.  Auch
eine redaktionelle Bearbeitung des Werks ist möglich.

Die Freigabemeldung kann vom Autor zurückgezogen werden, so lange das Werk noch
nicht freigegeben ist, sich also noch im Zustand „in Begutachtung“ befindet.

Ein freigegebenes Werk kann nicht verändert werden. 

\subsubsection{Löschen eines Werks}\label{werk.loeschen}

Es ist möglich ein \glossar{Werk} zu löschen. Dies sollte für bereits
veröffentlichte Werke der Ausnahmefall bleiben, kann aber aus verschiedenen
Gründen erforderlich sein:
\begin{itemize}
\item Die Auf"|forderung zum Entfernen eines Werks wird von außen an die
  Plattformbetreiber herangetragen, etwa bei der Verletzung von
  Urheberrechten.
\item Der \glossar{Autor} fordert die Plattformbetreiber auf, ein bereits
  veröffentlichtes Werk zu entfernen und dieser Auf"|forderung ist
  billigerweise nachzukommen.
\end{itemize}
Eine solche Auf"|forderung zum Löschen eines Werks ist grundsätzlich an einen
\glossar{Admin} zu richten, der das Weitere veranlasst.

Weiterhin können Werke gelöscht werden, die noch nicht veröffentlicht wurden:
\begin{itemize}
\item Das Werk wird während des Erstellngsprozesses vom \glossar{Autor} selbst
  verworfen.
\item Das Werk genügt nicht den Plattformanforderungen und wird von der
  Plattformredaktion verworfen. 
\end{itemize}
In jedem Fall wird dabei der Status des Werks auf „gelöscht“ gesetzt, das Werk
selbst und die zugehörigen Dateien aber nicht aus der Datenbasis gelöscht. Ein
Werk im Werkstatus „gelöscht“ ist weiter in der Plattform mit allen seinen
Informationen für einen \glossar{Admin} verfügbar, wird aber weder angezeigt
noch kann es vom (ehemaligen) Autor bearbeitet oder auch nur gesehen werden.

Vor dem \emph{Löschen bereits veröffentlichter Werke} prüft der Admin, welche
\glossar{Projekte} von der Löschung betroffen sind, welche Auswirkung das
Löschen auf diese Projekte hat, ob diese Projekte ggf. selbst zu löschen sind
oder modifiziert werden müssen, löst die entsprechenden Aufträge an die
\glossar{Projektleiter} aus und überwacht den Fortgang.

Beim Löschen eines Werks werden die Verweise auf dieses Werk aus allen
\glossar{Projekten} und nutzerspezifischen Bewertungen gelöscht.  Es verbleibt
allein eine rudimentäre standardisierte Information zum Werk im System.  Ein
solches Werk kann von einem \glossar{Admin} wieder reaktiviert werden.

\subsubsection{Entfernen eines gelöschten Werks aus der
  Plattform}\label{werk.entfernen} 

Ein gelöschtes Werk kann vollständig aus der Plattform entfernt werden.  Dies
kann nur durch einen Admin ausgeführt werden. Dabei werden alle relevanten
Werkteile im \glossar{PVM-Paketformat} in einer Datei gespeichert und danach
die Werkteile aus der Plattform komplett gelöscht.  Ein vollständig gelöschtes
Werk kann nur durch erneutes Einspielen der Einzelteile wieder reaktiviert
werden.

\subsection{Möglichkeiten für \glossar{Projektleiter}}

\subsubsection{Allgemeines}
 
Projekte können nur von \glossar{Projektleitern} erstellt werden.

Ein \glossar{User} wird \glossar{Projektleiter}, indem er dies bei einem
\glossar{Admin} beantragt und von diesem die erforderlichen Rechte zugeordnet
werden.  Ein User kann erst dann einen Projekt"|erstellungsprozess starten,
wenn er als Projektleiter registriert ist.

Projektleiter erstellen und verwalten \glossar{Projekte} im Rahmen ihrer
Möglichkeiten und Intentionen. Wenn der Projektleiter sein Projekt löscht, so
wird bei allen \glossar{Werken}, die bisher diesem Projekt zugeordnet waren,
lediglich die Projektzugehörigkeit gelöscht, nicht jedoch die Werke selbst.

\subsubsection{Projekt erstellen und bestätigen}

\begin{itemize}
\item Ein neues \glossar{Projekt} kann nur von einem \glossar{Projektleiter}
  erstellt werden.
\item Der Projektleiter reicht über die \emph{Projektformularseite} einen
  \emph{Antrag} auf ein neues Projekt ein, welcher (wenigstens) Projektkürzel,
  Titel, Projektfoto und \emph{Projektkonzept} umfasst. Aus den Metadaten
  werden Projektleiter und Einreichungsdatum ergänzt.

  Der Bezug zu \glossar{Projektangeboten} ist nur lose, neue Projekte müssen
  sich nicht auf eines der Projektangebote beziehen.

\item Ein solcher \emph{Projektantrag} wird von einem \glossar{Redakteur}
  begutachtet und bestätigt oder abgelehnt.  Die Entscheidung wird dem
  Einreicher per E-Mail mitgeteilt. 

\item Für einen bestätigten \emph{Projektantrag} legt der \glossar{Redakteur}
  die \glossar{Projektseite} an.  In der Durchführungsphase ist diese Seite
  nur über einen Direktlink zu erreichen.

  Die Seite wird erst nach Abschluss des Projekts als Kachel auf der Seite
  „Alle Projekte“ integriert und damit in die Menüstrukturen übernommen.

\item Nach dem Anlegen dieser Projektinfrastruktur in der Plattform wird der
  Projektleiter informiert und kann sein Projekt im Rahmen der
  \emph{Richtlinien für Projekte} umsetzen. 

\end{itemize}

Die Projektseite enthält drei Texte (Kurztext, Projektdokumentation,
Abschlusstext) sowie dokumentarische Bilder, welche vom Projektleiter
bereitgestellt und während des Projektverlaufs überarbeitet werden können.

\subsubsection{Teilnehmer eines Projekts}

Es gibt keine direkte Zuordnung von Teilnehmern zu einem Projekt. Eine solche
Zuordnung erfolgt nur über die \glossar{Werke} des jeweiligen \glossar{Autors},
die dieser für das Projekt bereitstellt und die vom \glossar{Projektleiter}
angenommen werden.

\subsubsection{Zuordnen von Werken zu einem Projekt}

Werke können nur bereits veröffentlichten, aber noch nicht archivierten
Projekten zugeordnet werden. 
\begin{itemize}
\item Einem \glossar{Projekt} können nur \glossar{Werke} zugeordnet werden, die
  bereits in den \glossar{Fundus} aufgenommen wurden und damit den allgemeinen
  Prozess der \glossar{Freischaltung} von Werken durchlaufen haben.

\item Ein \glossar{User} kann vom ihm verwaltete \glossar{Werke} (eigene oder
  Werke ihm zugeordneter Autorenprofile) für ein Projekt über einen Link auf
  der Werkseite vorschlagen.

\item Der \glossar{Projektleiter} entscheidet über die Aufnahme vorgeschlagener
  Werke in das Projekt. Dies erfolgt über eine Auswahlliste der vorgeschlagenen
  Werke im WP-Backend des Projekts.

\item Der Projektleiter kann Werke aus dem Projekt wieder entfernen.  Zuordnen
  und Entfernen eines Werks zu einem Projekt ändert dessen Zugehörigkeit zum
  \glossar{Fundus} nicht.

\item Ein Werk kann gleichzeitig in mehreren Projekten referenziert sein.
\end{itemize}

\subsubsection{Löschen von Projekten}

Das Löschen eines bereits veröffentlichten \glossar{Projekts} sollte nur in
Ausnahmefällen geschehen und ist durch den \glossar{Projektleiter} bei einem
\glossar{Admin} zu beantragen.

Beim Löschen des Projekts wird die \glossar{Projektseite} aus dem PVM entfernt
und alle Verbindungen von \glossar{Werken} zu diesem Projekt gelöscht.  Dies
hat keinen Einfluss auf die Zugehörigkeit eines \glossar{Werks} zum
\glossar{Fundus}.

Über das Löschen des Projekts hinausgehende Aktionen (Löschen von
\glossar{Usern} oder \glossar{Werken}) sind einzeln wie in den Abschnitten
\ref{werk.loeschen} und \ref{nutzer.loeschen} beschrieben zu beantragen und zu
prozessieren.

\subsection{Möglichkeiten für \glossar{Redakteure}}

Aufgabe der Redakteure ist die \glossar{Freigabe} eingereichter \glossar{Werke}
sowie die Bearbeitung von gemeldeten Verstößen.  

\paragraph{Freigabe eingereichter Werke:} 
Dem Redakteur wird dazu im WP-Backend eine Liste aller zur Freigabe anstehenden
\glossar{Werke} angezeigt.  Zur Prüfung eines dieser Werke kann der Redakteur
das Werk (dieses befindet sich dabei im Werkstatus „in Begutachtung“) sich
anzeigen lassen und auch redaktionell bearbeiten.

Im Ergebnis der Prüfung kann der Redakteur 
\begin{itemize}
\item das Werk freigeben und damit in den \glossar{Fundus} übernehmen
  (Werkstatus „freigegeben“),
\item das Werk dem Autor zur Überarbeitung zurückgeben (Werkstatus „in
  Bearbeitung“),
\item das Werk löschen (Werkstatus „gelöscht“).
\end{itemize}
Der \glossar{Redakteur} informiert den \glossar{Autor} über das Ergebnis der
Prüfung.

\paragraph{Gemeldete Werke:} 
\glossar{User} sehen zu jedem \glossar{Werk} auf dessen \emph{Werkseite} einen
Link „Verstoß melden“.  Dieser dient dazu, \glossar{Redakteure} auf bereits
veröffentlichte Werke aufmerksam zu machen, die aus verschiedenen Gründen den
Standards des PVM nicht gerecht werden.  Es kann zum Beispiel vorkommen, dass
sich ein User absichtlich oder unwissentlich nicht an die AGB hält oder
Urheber-Rechte verletzt wurden. Um dagegen vorgehen zu können existiert die
Melde-Funktion.

Über diesen Link wird der \glossar{User} zu einem Formular weitergeleitet, über
das er an die \glossar{Redakteure} innerhalb der Website eine Nachricht senden
kann. Nach dem Abschicken dieser Nachricht wird im Backend unter „Gemeldete
Werke“ ein neuer Eintrag angelegt, der von den Redakteuren eingesehen und
bearbeitet werden kann. Ein Redakteur kann das entsprechende \glossar{Werk}
\begin{itemize}
\item wieder „entwarnen“ (also die Meldung löschen), 
\item zur Beratung zurückziehen (Werkzustand „in Begutachtung“),
\item an den Autor zur Überarbeitung zurückgeben (Werkzustand „in Bearbeitung“)
\item oder das Werk löschen (Werkzustand „gelöscht“). 
\end{itemize}
Beim Löschen eines Werks sind die im Abschnitt \ref{werk.loeschen}
beschriebenen Folgen zu beachten.

Gemeldete Werke bleiben bis zu einer redaktionellen Entscheidung sichtbar,
werden also nicht präventiv aus der \glossar{Dauerausstellung} entfernt.
  
\section{Konzeptionelle und Design-Fragen}

\subsection{Design}

Der Einstieg in die \glossar{Plattform} erfolgt über die \emph{Startseite}.
Weiter gibt es 
\begin{itemize}\itemsep0pt
\item eine \emph{Seite „Alle Werke“} mit Such- und Filterfunktion, auf der alle
  \glossar{Werke} als Kacheln angezeigt werden,
\item für jeden \glossar{Autor} eine \glossar{Profilseite}, 
\item für jedes \glossar{Werk} eine \emph{Werkseite},
\item eine \emph{Seite „Alle Projekte“}, auf der alle (abgeschlossenen)
  \glossar{Projekte} als Kacheln angezeigt werden, 
\item für jedes \glossar{Projekt} eine \emph{Projektseite}
\item eine \emph{Seite „Alle Projektangebote“}, auf der alle
  \glossar{Projektangebote} als Kacheln angezeigt werden, sowie
\item für jedes \glossar{Projektangebot} eine \emph{Projektangebotsseite}. 
\end{itemize}
Zur Einheitlichkeit der Darstellung ist für jede dieser Seiten
bzw.\ Seitentypen im PVM-Theme jeweils ein eigenes \emph{Seitentemplate}
angelegt.

Daneben gibt es weitere Seiten mit erklärendem Charakter -- „About“,
„Tutorium“, „Impressum“ -- sowie die WP-Standardfunktionen \emph{Login},
\emph{Registrierung} und \emph{Passwort-Reset}, die in die Menüführung
eingeordnet sind.

Der Menüpunkt „Inspiration“ wird derzeit als Blog von Usern in der Rolle
\glossar{Blogautor} geführt und ist perspektivisch ebenfalls mit einem eigenen
Seitentemplate zu unterlegen, das einen erklärenden Einstiegstext mit der
Anzeige einer Liste von Blogeinträgen verbindet. 

\subsection{Menüführung}

Die Menüführung erfolgt primär in zwei Ebenen über ausklappbare Links in der
Menüleiste.

Um die eigentliche Menüführung schlank zu halten, erfolgt die Menüführung über
Klassen mit einer größeren Menge von Instanzen (Werke, Projekte) über eine
Übersichtsseite wie \emph{Alle Werke}, auf der alle Instanzen als Bildkacheln
dargestellt sind und Filterfunktionen zur Verfügung stehen, um die Auswahl
einzuschränken. Die Filterfunktionen werden dabei über ein Widget im
Navigationsteil im linken Panel der jeweiligen Seite eingebunden.

Einzelne Seiten sind auch über Direktlinks zu erreichen.  Die Projektseiten
noch nicht abgeschlossener Projekte sind \emph{nur} über ihren Direktlink zu
erreichen.

\subsection{Frontend und Backend}

Wordpress unterscheidet zwischen dem Präsentationsbereich (Frontend) und dem
Administrationsbereich (Backend).  Während das Frontend das „Schaufenster“ der
Anwendung ist, kann das Backend grundsätzlich nur von angemeldeten Benutzern
per Navigation über die nach der erfolgreichen Anmeldung am oberen Rand
sichtbare schwarze WP-Standard-Leiste erreicht werden.  Das „Look and Feel“ des
Frontends ist in dem Umfang einheitlich wie dies das Design der Plattform
vorsieht, das Backend stellt rollenabhängig Funktionalität in verschiedenem
Umfang bereit.

Das Backend wird für die Rollen \glossar{Redakteur} und \glossar{Projektleiter}
um PVM-spezifische Funktionen erweitert.  User in diesen Rollen müssen also
hinreichend mit dem allgemeinen „Look and Feel“ des WP-Backends vertraut sein.

Das Management von \glossar{Werken} erfolgt komplett über Formulare im Frontend
der \glossar{Plattform}, so dass \glossar{Autoren} nicht im Backend arbeiten
müssen.

\subsection{Seitenaufbau der Plattform}

\subsubsection{Startseite}

Die \emph{Startseite} ist die Einstiegsseite in die Plattform.  Dort werden
einerseits die verfügbaren Merkmale und andererseits die im Fundus vorhandenen
Werke aufgelistet bzw.\ präsentiert.

\subsubsection{Übersichtsseite „Alle Werke}
Auf der \emph{Übersichtsseite „Alle Werke“} werden alle vorhandenen
\glossar{Werke} gelistet und verschiedene Such- und Filterfunktionen
angeboten, um die Zahl der angezeigten \glossar{Werke} zu beschränken.  Jedes
Werk wird durch eine Kachel repräsentiert.

\subsubsection{Übersichtsseite „Alle Projekte“}
Auf der \emph{Übersichtsseite „Alle Projekte“} werden alle abgeschlossenen
\glossar{Projekte} gelistet.  Jedes abgeschlossene Projekt wird durch eine
Kachel repräsentiert.

\subsubsection{Profilseite}
Jeder \glossar{Autor} hat eine Profilseite, auf der Standardinformationen zum
Autor angezeigt und seine Werke gelistet werden. Ist der Autor zugleich
\glossar{User}, so werden weitere User-bezogene Informationen wie etwa die
eigenen Favoriten und eine Übersicht der Projekte des Autors angezeigt. Über
die Profilseite hat der User im angemeldeten Zustand zugleich Zugang zu
verschiedenen weiteren Funktionen im Rahmen der ihm übertragenen Rechte.

Persistente Informationen müssen in die Tabelle \texttt{pvmkit\_authors}
eingetragen werden, dort ist derzeit nur der \texttt{foaf:name} sowie eine
Spalte \texttt{location} hinterlegt.

Über die Profilseite sind im angemeldeten Zustand bei entsprechenden Rechten
die Funktionen „Neues Werk erstellen“ und „Benutzerrechte bearbeiten“
erreichbar.

\subsubsection{Projektseite}
Seite, auf der ein einzelnes \glossar{Projekt} angezeigt wird.  Diese Seite
wird erst auf der Seite „Alle Projekte“ verlinkt, wenn das Projekt
abgeschlossen ist. 

\subsubsection{Werkseite}
Seite, auf der ein einzelnes \glossar{Werk} angezeigt wird.  In der Sidebar
werden die \glossar{Merkmale} des Werks angezeigt, im unteren Teil des
Hauptpanels die \glossar{Bewertungen} durch \glossar{User}.  

Im angemeldeten Zustand hat der User weiter die Möglichkeit, eine eigene
Bewertung abzugeben oder zu ändern.
  
\section{Offene Anforderungen} 

\begin{itemize}
\item Import von plattformweit vorgegebenen Informationen (etwa Merkmale, Orte,
  Bewertungen) beim Aufsetzen der Plattform. Dies ist im Zuge der Präzisierung
  des Relokationskonzepts noch genauer zu spezifizieren.

  Diese Informationen liegen als RDF-Dateien (\emph{Bewertungen.ttl},
  \emph{Merkmale.ttl} und \emph{Orte.ttl}\/) für Bewertungen, Merkmale und für
  Orte im Verzeichnis \texttt{Daten/Vorgaben} vor und müssen beim Aufsetzen
  der Plattform aus diesen Quellen eingelesen werden.

  Zu klären ist weiter, wie Modifikation dieser Informationen im laufenden
  Betrieb persistiert werden. Dazu muss die Datenbank \emph{und} die externe
  Quelle aktualisiert werden. Alternativ kann die Möglichkeit von RDF-Exporten
  entsprechender Datenbanktabellen vorgesehen werden.

\item Die genaue Realisierung der Funktionalität „Zuletzt besuchtes Werk“ im
  Gestaltungsentwurf muss noch besprochen und priorisiert werden.
\end{itemize}

\section{Glossar} 

\paragraph{Administrator:} 
Standard-WP-Rolle, die auf alle Daten und Funktionen innerhalb des Systems
zugreifen kann, Nutzeranmeldungen freischaltet, Nutzern Rollen zuweist und die
Plattform konfiguriert.  Sie ist allen anderen Rollen in diesem Sinne
übergeordnet.

\paragraph{Autor:} 
\glossar{User} mit eigenen \glossar{Werken}.  Werke können einen Autor haben,
der nicht mehr als User an der Plattform aktiv ist. Neue Werke können nur von
aktiven Usern eingebracht werden. Jedes Werk besitzt einen eindeutigen Autor.
Ein \glossar{User} kann für \emph{mehrere} Autorenprofile zuständig sein, wenn
für ihn ein solches Recht freigeschaltet ist.

\paragraph{Besucher:} 
Als Besucher wird eine nicht eingeloggte Person bezeichnet, die den
öffentlichen Bereich des PVM nutzt.

\paragraph{Bewertungssystem:} 
System plattformweit fest vorgegebener Begriffe mit bildlichen Icons, das
\glossar{User} zur Bewertung von \glossar{Werken} einsetzen können.

Die Überwachung freier Kommentare ist mit den verfügbaren Personalressourcen
nicht zu bewältigen, deshalb wird es (wenigstens in dieser Ausbaustufe des PVM)
keine Möglichkeit zu freien Kommentaren und auch kein Forum geben.  

\paragraph{Bild-Teil:}
Spezielle Art von \glossar{Werkteil}, das eine Bilddatei im png-Format enthält.
Um ein einheitliches Erscheinungsbild innerhalb der \glossar{Plattform} zu
garantieren, muss ein Bild-Teil spezielle \emph{Designvorgaben} erfüllen.

\paragraph{Blogautor:} 
Ausgewählte \glossar{User} in der WP-Rolle „Autor“, die den Blog unter dem
Menüpunkt „Inspiration“ führen. 

\paragraph{Dauerausstellung:} 
Präsentation aller \glossar{Pakete} aus dem \glossar{Fundus} im (näher zu
definierenden) \emph{Allgemeinkontext} des PVM-Projekts.

\paragraph{Einreichung:} 
Ein \glossar{Urheber} hat ein \glossar{Werk} erstellt und meldet dieses zur
\glossar{Freigabe}.

\paragraph{Formularseite:} 
Webseite, über welche die Einzelteile eines \glossar{Pakets} zusammengetragen
und gespeichert werden kön"|nen.  

\paragraph{Freigabe:} 
Prüfprozess, der nach der \glossar{Einreichung} eines \glossar{Werks} durch den
\glossar{Autor} einsetzt. Ein \glossar{Redakteur} entscheidet über Freigabe
oder Zurückweisung des eingereichten {Werks}.

\paragraph{Fundus:} 
Sammlung aller aktuell veröffentlichten \glossar{Werke}. 

\paragraph{Merkmal:} 
Plattformweit vorgegebene Paare (Merkmalsart, Ausprägung), die vom
\glossar{Autor} zur Charakterisierung eines \glossar{Werks} verwendet werden
können.  Die möglichen Merkmale sind als \emph{morphologische Tabelle} in Form
einer Merkmalsontologie vorgegeben.

\paragraph{Paket:} 
Grundlegende Einheit, in der ein \glossar{Werk} zusammengefasst wird, das 
\begin{itemize}\itemsep0pt
\item ein Container verschiedener \glossar{Werkteile} ist,
\item einer festen Struktur folgt, die als \glossar{Paket-Metainformation} in
  digitaler Form Teil des Pakets ist,  
\item die im \glossar{PVM-Paketformat} zwischen Plattformen ausgetauscht werden
  kann.
\end{itemize}

\paragraph{Plattform:} 
Synonym für die zunächst unter der Bezeichnung \glossar{Museum} gefasste
Erweiterung des Begriffs \glossar{Ausstellung} um eine
didaktisch-konzeptionelle Dimension. 

\paragraph{Profilseite:} 
Jeder \glossar{Autor} hat eine eigene Profilseite, auf der u.a. die von diesem
Autor eingebrachten Weke gelistet sind.  Auf der Profilseites eines Autors,
der zugleich \glossar{User} ist, werden weitere personenbezogene Informationen
angezeigt wie etwa die von diesem User ausgewählten Favoriten. 

Wird der User gelöscht, so bleibt die Profilseite erhalten, es werden danach
aber keine userspezifischen Präferenzen mehr angezeigt.  Eine Autorenseite kann
nur zusammen mit den Werken dieses Autors und nur von einem \glossar{Admin}
gelöscht werden.

\paragraph{Projekt:} 
In einem \glossar{Projekt} können Lehrer \emph{eigene} didaktische
Zielstellungen mit den Möglichkeiten unserer \glossar{Plattform} umsetzen oder
eines unserer \glossar{Projektangebote} für die eigene Zielgruppe nutzen und
dazu entsprechend modifizieren.

\paragraph{Projektangebot:} 
Spezifischer didaktischer Rahmen zur Nutzung der Plattform. Im Rahmen des PVM
werden Projektangebote systematisch entwickelt und dazu \glossar{Sets} aus
Materialien bereitgestellt.

\paragraph{Projektleiter:} 
Erweiterung einer WP-Rolle um das Recht, Projekte anzulegen und zu verwalten.
Jeder \glossar{User} kann von einem \glossar{Admin} zum Projektleiter bestimmt
werden. Jedes \glossar{Projekt} verfügt über einen Projektleiter.

\paragraph{PVM-Paketformat:}
Spezifischen Binding der \glossar{PVM-Paketstruktur}, um ein \glossar{Paket}
als Digitalisat in einer einzelnen Datei zu speichern.

\paragraph{PVM-Paketstruktur:}
Innere Struktur eines Pakets als in RDF beschriebenes \emph{Datenmodell} samt
eines spezifischen Bindings an das \glossar{PVM-Paketformat}.  Ein Paket
enthält
\begin{itemize}
\item Autor, Entstehungszeitpunkt, Ort des Autors, Titel als
  Paket-Metainforma"|tion,
\item ein vom \glossar{Autor} selbst erstelltes oder urheberrechtlich zulässig
  nachgenutztes \emph{Titelbild} als \glossar{Bild-Teil},
\item eine vom \glossar{Autor} selbst erstellte Beschreibung als
  \glossar{Text-Teil},
\item eine bildnerische Interpretation als weiteren \glossar{Bild-Teil},
\item zugeordnete Merkmale als Menge von URIs aus der
  \glossar{Merkmalsontologie} als Teil der \glossar{Paket-Metainfomation}.
\end{itemize}

\paragraph{Redakteur:} 
Erweiterung der WP-Rolle „Redakteur“ um die Aufgaben des
\glossar{Freischaltens} von \glossar{Werken} und der Bearbeitung von Meldungen.

\paragraph{Set:} 
Menge von Download-Materialien zu einem \glossar{Projektangebot}.  Einzelne
Materialien können zu mehreren Projektangeboten gehören.  Die Materialien sind
einheitlich im Uploadbereich abgelegt, die Zugehörigkeit zu einzelnen
Projektangeboten wird durch eine Angebotsbeschreibung im RDF-Format aufgelöst.

\paragraph{Text-Teil:}
Spezielle Art von \glossar{Werkteil}, das eine Textdatei als valides
HTML-Fragment enthält.  Um ein einheitliches Erscheinungsbild innerhalb der
\glossar{Plattform} zu garantieren, muss ein Text-Teil spezielle
\emph{Designvorgaben} erfüllen.

\paragraph{User:} 
Registrierter und angemeldeter Nutzer der \glossar{Plattform}, der damit auch
die AGB akzeptiert hat. Einem User wird ein Autorenprofil zugeordnet und damit
eine eigene \glossar{Profilseite}.

\paragraph{Werk:} 
Ein \glossar{Paket} speziell unter urheberrechtlichen sowie
plattform-technischen Gesichtspunkten.

\paragraph{Werkteil:}
Bestandteil eines \glossar{Werks}, das eine physische Repräsentation als Datei
hat und so in ein \glossar{Paket} aufgenommen werden kann. Werkteile sind
einzeln nicht direkt zugänglich.  \glossar{Pakete} enthalten neben den
Werkteilen auch \emph{Beschreibungen} dieser Werkteile als Metainformationen.

Es gibt \glossar{Bild-Teile} und \glossar{Text-Teile}.

\end{document}
