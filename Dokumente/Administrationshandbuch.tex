\documentclass[11pt,a4paper]{article}
\usepackage{a4wide,ngerman,url}
\usepackage[utf8]{inputenc}

\parindent0pt
\parskip3pt

\title{Installation und Administration der PVM-Plattform}
\author{Hans-Gert Gräbe, Sebastian Günther, Alexander Lieder}
\date{Version vom 18. September 2016}

\begin{document}
\maketitle
\begin{quote}
Dieses Dokument beschreibt, welche personellen Voraussetzungen für den
Plattformbetrieb erforderlich sind und wie eine Plattform-Instanz auf der Basis
einer Standard-Wordpress-Installation auf einem entsprechend konfigurierten
Web-Server Schritt für Schritt aufgebaut und verwaltet werden kann.
\end{quote}
\tableofcontents{}\newpage{}


\section{Allgemeines}

Zum Betrieb der Wordpress-Produktivinstanz der PVM-Plattform sind folgende
Rollen zu besetzen:
\begin{itemize}
\item \emph{Site-Admin} --- zuständig für die Installation und Aktualisierung
  der grundlegenden Software auf dem Rechner, das Basis-Backup sowie Kontakt
  bei Rechnerausfällen und Neustart.
\item \emph{Portal-Admin} -- Aktualisierung der Wordpress-Instanz auf eine
  neue Version, Installation und Aktualisierung von Plugins und des
  PVM-eigenen Themas, Upload-Verwaltung. Ein Portal-Admin hat Zugriff auf das
  Installationsverzeichnis auf dem Server.
\item \emph{WP-Admin} --- zuständig für konzeptionelle Fragen sowie die
  Konfiguration der verschiedenen Anwendungen, Plugins und Funktionen über die
  Admin-Konfigurations-Oberfläche von Wordpress. Ein WP-Admin hat keinen
  Zugriff auf das Installationsverzeichnis auf dem Server.
\item \emph{WP-Redakteure} --- zuständig für die Inhalte der Plattform. 
\end{itemize}

\section{Installation und Verwaltung der Wordpress Site}

\subsection{Initialer Aufbau der Plattform}

Die Plattform wird initial über die Standard-Installationsprozesse von
Wordpress aus den im PVM-Release mitgelieferten Daten von einem
\emph{Portal-Admin} über folgende Schritte aufgebaut:
\begin{enumerate}
\item Aufsetzen einer Standard-Installation von Wordpress\footnote{Siehe etwa
  \url{http://wpde.org/installation/}. }. Wenn dies eine längerfristig laufende
  Website werden soll, wird empfohlen, die üblichen allgemeinen\footnote{Siehe
    etwa \url{http://httpd.apache.org/docs/2.4/misc/securitytips.html}.  } und
  Wordpress-spezifischen Sicherheitshinweise\footnote{Siehe etwa
    \url{https://wordpress.org/about/security/}.  } zu beachten.
\item Einstellen von \emph{Permalinks} unter
  \emph{Einstellungen\,$\to$\,Permalinks}.

Wordpress versucht dabei, im Installationsverzeichnis eine \texttt{.htaccess}
Datei anzulegen, welche das Rewriting der Links von den absoluten
Seitenadressen auf die Post und Page Id's umsetzt. Kann der Webserver diese
Datei nicht schreiben, so muss der Inhalt (siehe dazu die Anzeige im Menü
\texttt{Einstellungen\,$\to$\,Permalinks}) manuell auf den Server übertragen
werden.

\item Einspielen, Aktivieren und Konfigurieren der verwendeten
  Standard-Plugins aus dem Release-Verzeichnis \emph{Plugins-and-Themes}.

Die Plugins müssen dazu ins Verzeichnis \texttt{wp-content/plugins/} kopiert
und dort ausgepackt werden.

\item Einspielen des im Release mitgelieferten \emph{WP-Backups}. Dieses
  Backup ist ein komplettes Backup aller \emph{Seiten} der Plattform, das mit
  der unter \emph{Werkzeuge\,$\to$\,Datenexport} für WP-Admins verfügbaren
  Exportfunktion erstellt wurde.

Für den Import muss das (mitgelieferte) Plugin \emph{wordpress-importer}
installiert und ein Verzeichnis \texttt{wp-content/uploads/} für den Webserver
schreibbar sein.

\end{enumerate}
Danach kann ein \emph{WP-Admin} die Menü- und Widgetstruktur über das
WP-Admin-Interface einrichten. Folgende Modifikationen sollten vorgenommen
werden.
\begin{enumerate}
\item Unter \texttt{Einstellungen\,$\to$\,Diskussion} Kommentare abschalten.
\item Einstellungen des Plugins \texttt{wp-members} anpassen: Die Seiten
  „Login“, „Registrierung“ und „Profil“ müssen den entsprechenden Funktionen
  im Tab „WP-Members Optionen“ zugeordnet sowie die Registrierungsfelder
  konfiguriert werden.
\item Unter \texttt{Design\,$\to$\,Menüs} die Menüs für den Header (Primary
  Menu), das ProjectMenu und den Footer (Footer Navigation Menu) anlegen.
\item Seiten für die Menüs anlegen und dem jeweiligen Menü zuordnen
\item Damit innerhalb der Seiten die korrekte Sidebar angezeigt wird, muss
  diese in den Einstellungen der jeweiligen Seiten ausgewählt werden.  Soll
  die Sidebar leer bleiben, reicht es nicht, eine leere Sidebar auszuwählen,
  da diese von Wordpress ignoriert wird. Die Sidebar muss mindestens ein
  Widget enthalten. Um dieses Problem zu umgehen, kann beispielsweise ein
  leeres Widget angelegt und der Sidebar zugewiesen werden.
\item Die Spezialseiten (Alle Werke, Einzelnes Werk, Benutzerprofil etc.)
  anlegen.
\item Den Spezialseiten das entsprechende Template zuordnen und \textbf{den
  Permalink anpassen.}\footnote{HGG: Wird das nicht bereits durch den WP-Dump
  realisiert?} 

\begin{enumerate}
\item Alle Werke: Homepage
\item Einzelnes Werk: Homepage/single\_package
\item Profilseite: Homepage/single\_user
\item Benutzerwerke: Homepage/profile\_package\_list
\end{enumerate}
\item Unter \texttt{Design\,$\to$\,Widgets} den Spezialsidebars die
  entsprechenden Widgets zuordnen:

\begin{enumerate}
\item All Packages Sidebar: Suche, Merkmale
\item Single Package Sidebar: Werkmeta
\end{enumerate}
\end{enumerate}

\section{Backupanforderungen und Upgrade der Plattform}

Der Zustand der Plattform lässt sich aus regelmäßigen Backups des
Installationsverzeichnisses, der Datenbank sowie folgender weiterer Daten (zu
ergänzen) rekonstruieren. Ein entsprechender \emph{Backup-Prozess} muss durch
den \emph{Site-Admin} spezifiziert und regelmäßig ausgeführt werden.

Ein \emph{Upgrade der Plattform} kann in der üblichen Weise durch einen
\emph{Portal-Admin} erfolgen durch einzelne oder alle der folgenden
Operationen
\begin{itemize}
\item Upgrade der WP-Basisinstallation, 
\item Upgrade einzelner Plugins, 
\item Upgrade des PVM-Themes. 
\end{itemize}
Dabei sind die üblichen Upgrade-Regeln\footnote{Siehe etwa
  \url{https://codex.wordpress.org/UpdatingWordPress}. } von Wordpress
anzuwenden. Die Möglichkeiten eines automatischen Background Updates wurden
nicht getestet.


\section{Beschreibung der aktuell verwendeten Wordpress-Instanzen}


\subsection{Grundsätzlicher Aufbau}

Die \emph{Wordpress-Produktivinstanz} ist im Verzeichnis
\texttt{public\_html/wp} auf der Basis der Word"|press-Version 4.6.1 (Stand:
18.09.2016) als SVN-Installation ausgerollt. Der Apache-Server ist so
konfiguriert\footnote{Siehe \texttt{/etc/apache2/vhosts.d/pvm.conf} }, dass
dieses Verzeichnis auch als \texttt{pvm.uni-leipzig.de} ausgeliefert wird. Site
und Home URL der Produktivinstanz sind entsprechend eingestellt.

Die \emph{Wordpress-Entwicklungsinstanz} ist im Verzeichnis
\texttt{public\_html/pvmdevelop} auf der Basis der Wordpress-Version 4.5.3
(Stand: 18.09.2016) als SVN-Installation ausgerollt.

Beide Instanzen verwenden dieselbe Datenbank mit verschiedenen
Tabellen-Präfixen, die Produktivinstanz verwendet das Präfix \texttt{pvm\_},
die Entwicklungsinstanz das Präfix \texttt{pvmdevelop\_}\,. In beiden Instanzen
ist \texttt{WP\_DEBUG} eingeschaltet.

In beiden Instanzen ist die Permalink-Option aktiviert. Zum Funktionieren der
\emph{Permalinks} ist im jeweiligen Installationsverzeichnis eine
\texttt{.htaccess} Datei angelegt, welche das Rewriting der Links von den
absoluten Seitenadressen auf die Post und Page Id's umsetzt.

Im Verzeichnis \texttt{wp-content/uploads/pvm/} sind die Werkdaten
gespeichert. 

\subsection{Plugins}

Plugins enthalten spezifische funktionale Erweiterungen der
Standard-WP-Plattform.  Sie werden als zip-Datei verteilt, müssen im
Verzeichnis \texttt{wp-content/plugins/} als Unterverzeichnisse ausgerollt und
dann über das Admin-Backend aktiviert werden.

Zum Betrieb sind die folgenden Plugins installiert (Stand 17.01.2016): 
\begin{itemize}
\item \texttt{wp-members} -- Erweiterung der Nutzerverwaltung von Wordpress.
\item \texttt{wp-mail-smtp} -- Anpassung der E-Mail-Adresse der Plattform an
  die lokalen Gegebenheiten des E-Mail-Versands auf dem Server.
\item \texttt{wordpress-importer} -- Import der Seitenstruktur aus einem
  XML-basierten Dump der Seitenstruktur einer anderen Plattform.
\item \texttt{pvmkit} -- Plugin mit PVM-spezifischen Funktionalitäten.
\item \texttt{pvm2rdf} -- Plugin mit Shortcodes zum Export der Werk-Metadaten
  im RDF-Format (experimentell, für den Betrieb aktuell nicht erforderlich).
\end{itemize}
Alle benötigten Plugins liegen als zip-Dateien im Repo-Verzeichnis
\texttt{Plugins-and-Themes} und können von dort durch einen
\emph{Portal-Admin} ausgerollt bzw.\ aktualisiert und im Backend unter
„Plugins installieren“ aktiviert werden.

\subsection{Themes}

Themes sind die Basis, um das spezifische „Look and Feel“ einer Website
einzustellen. Themes liefern Grund-Design-Funktionalitäten wie Menüstrukturen,
Seitentemplates, CSS Style sheets, Short Codes usw., die an verschiedenen
Stellen im Admin-Backend eingebunden sind und dem WP-Admin eine
Basiskonfiguration der Plattform ermöglichen.

Auch Themes werden als zip-Datei verteilt und müssen durch einen
\emph{Portal-Admin} im Verzeichnis \texttt{wp-content/themes/} als
Unterverzeichnis ausgerollt werden. Zur gleichen Zeit kann immer nur ein Theme
aktiviert sein.

Für die PVM-Plattform wurde das Child Theme \texttt{esteem\_pvm\_child} zum
Basisthema \emph{esteem} entwickelt. Das Basistheme ist ein Theme, das ein
spezielles „Look and Feel“ umsetzt und von einer größeren Community aktiv
weiterentwickelt wird. Im Child Theme müssen dann nur Modifikationen gegenüber
diesem Basistheme definiert werden, so dass künftiger eigener
Weiterentwicklungsaufwand reduziert werden kann.

\section{Weitere Hinweise}

\paragraph{Wordpress aus dem WP-SVN installieren:}

Wechsle in das Ziel-Verzeichnis und (abschließenden Punkt nicht vergessen --
aktuelle Versionsnummer einsetzen)
\begin{quote}
\texttt{svn co http://core.svn.wordpress.org/tags/4.4.1 . }
\end{quote}
Danach \texttt{wp-config-sample.php} nach \texttt{wp-config.php} kopieren, die
lokalen Daten"|bank-Zu"|gangs"|daten eintragen und weiter wie unter
\begin{quote}
{\small\url{http://codex.wordpress.org/InstallingWordPress} und}\\
{\small\url{https://codex.wordpress.org/Installing/UpdatingWordPresswithSubversion}}
\end{quote}
beschrieben. Insbesondere sollte das Auslesen des \texttt{.svn}-Verzeichnisses
über entsprechende Einträge in der \texttt{.htaccess}-Datei wie dort
beschrieben verhindert werden.


\paragraph{SVN auf neue WP-Version aktualisieren:}

Deaktiviere alle Plugins. Wechsle ins Installationsverzeichnis und (sw wie
switch -- aktuelle Versionsnummer einsetzen)
\begin{quote}
\texttt{svn sw http://core.svn.wordpress.org/tags/4.4.2 . }
\end{quote}

\paragraph{Debug Modus:}

Dieser kann in der Datei \texttt{wp-config.php} eingeschaltet werden.  Nach
der PHP-Konstanten \texttt{WP\_DEBUG} suchen.


\paragraph{Wordpress auf Deutsch umstellen:}

Dazu müssen unter \texttt{wp-content/languages} die deutschen Sprachdateien
eingespielt sein. Dann kann ein WP-Admin im Menü
\emph{Einstellungen\,$\to$\,Allgemeines} die Sprache ändern.
\end{document}
